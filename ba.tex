% This is samplepaper.tex, a sample chapter demonstrating the
% LLNCS macro package for Springer Computer Science proceedings;
% Version 2.20 of 2017/10/04
%
\documentclass[runningheads]{llncs}
%
\usepackage{minted}
\usepackage{pgf}
\usepackage{import}
\usepackage{layouts}
\usepackage{graphicx}
\usepackage{tikz}
\usetikzlibrary{shapes.geometric, arrows}
\usepackage{amsmath}
\usepackage{adjustbox}
\usepackage{placeins}
\usepackage{float}
\usepackage{xspace}
\usepackage{array}
\usepackage{tabularx,booktabs}\usepackage[hidelinks]{hyperref}
\usepackage{cleveref}
\usepackage{afterpage}
\usepackage{svg}
\usepackage{xcolor}
\usepackage{appendix}
\usepackage[nottoc]{tocbibind}
\usepackage[ngerman,english]{babel}
\newcolumntype{Y}{>{\centering\arraybackslash}X}


% Used for displaying a sample figure. If possible, figure files should
% be included in EPS format.
%
% If you use the hyperref package, please uncomment the following line
% to display URLs in blue roman font according to Springer's eBook style:
% \renewcommand\UrlFont{\color{blue}\rmfamily}

\renewcommand\#{\protect\scalebox{0.8}{\protect\raisebox{0.4ex}{\char"0023}}}
\definecolor{darkgrey}{RGB}{100,100,100} 
\newcommand{\sort}{SAMtools \texttt{sort}\xspace}
\newcommand{\points}{Data points: Median, Error bars: fastest and slowest of 3 runs. }
\newcommand{\parents}{Numbers in parentheses indicate the compression level. }
\newcommand{\threads}{One thread is single threaded computation; for a higher thread count, \sort uses the threads in addition to the main thread. }
\interfootnotelinepenalty=100000


\begin{document}

%
\title{Faster Sorting of Aligned DNA-Read Files}
%
%\titlerunning{Abbreviated paper title}
% If the paper title is too long for the running head, you can set
% an abbreviated paper title here
%
\author{Dominik Siebelt}
% Second Author\inst{2,3}\orcidID{1111-2222-3333-4444} \and
% Third Author\inst{3}\orcidID{2222--3333-4444-5555}}
%
\authorrunning{D.\@ Siebelt}
% First names are abbreviated in the running head.
% If there are more than two authors, 'et al.' is used.
%
\institute{Karlsruhe Institute of Technology}
% Springer Heidelberg, Tiergartenstr. 17, 69121 Heidelberg, Germany
% \email{lncs@springer.com}\\
% \url{http://www.springer.com/gp/computer-science/lncs} \and
% ABC Institute, Rupert-Karls-University Heidelberg, Heidelberg, Germany\\
% \email{\{abc,lncs\}@uni-heidelberg.de}}
%
\maketitle              % typeset the header of the contribution
%
\begin{abstract}
In the analysis of DNA sequencing data for finding disease causing mutations, to understand evolutionary relationships between species, and to find variants, DNA-Reads are compared to a reference genome, which is a representative example for a set of genes of a species. Sorting these aligned DNA-Reads by their position within the reference sequence is a crucial step in many of these downstream analyses. \sort, a widely used tool, performs external memory sorting of aligned DNA-Reads stored in the BAM format (Binary Alignment Map). This format allows for compressed storage of alignment data. \sort provides the most comprehensive set of features while exhibiting demonstrably faster execution times than its open source alternatives.  In this work, we analyze \sort for sorting BAM files and propose methods to reduce its runtime. We divide the analysis into three parts: management of temporary files, compression, and input/output (IO).
For the management of temporary files, we find that the maximum number of temporary files \sort can open concurrently is lower than the maximum number of open files permitted by the operating system. This results in an unnecessarily high number of merges of temporary files into larger temporary files, introducing overhead as \sort performs extra write and compression operations. To overcome this, we propose a dynamic limit for the number of temporary files, adapting to the operating system's soft limit for open files.
For compression, we test seven different libraries for compatible compression and a range of compression levels, identifying options that offer faster compression and result in a speedup of up to five times in single-threaded execution of \sort.
For IO, we demonstrate that a minimal level of compression avoids IO overhead, thereby reducing the runtime of \sort compared to uncompressed output. However, we also show that uncompressed output can be used in the pipelining of SAMtools commands to reduce the runtime of subsequent SAMtools commands.
Our proposed modifications to \sort and user behavior have the potential to achieve speedups of up to 6. This represents an important contribution to the field of bioinformatics, considering the widespread adoption of \sort evidenced by its over 5,000 citations and over 5.1 million downloads through Bioconda.
\end{abstract}

\begin{otherlanguage}{ngerman}
\begin{abstract}
    
\end{abstract}
\end{otherlanguage}
%
\setcounter{tocdepth}{2}
\tableofcontents
\newpage
%
\section{Introduction}
To enable faster downstream analysis, aligned DNA-Read Files are sorted.

\subsection{SAM and BAM files}
A Sequence Alignment/Map (SAM) as specified by Li et al. \cite{samformat} is used to store the alignment of sequences against reference sequences. It consists of a Header Section and an Alignment Section. The Header Section contains meta information such as the format version or the sorting of the content and a dictionary of the reference sequences, whereas the Alignment Section contains aligned segments with alignment information and meta information such as the read quality. Here, a segment is a continuous sequence or subsequence of a raw DNA read. \\
The alignment information mainly consists of the ID of the reference sequence the alignment is mapped to, the position where the alignment starts in the reference sequence and a CIGAR string providing the alignment at this position. The CIGAR String as specified in \cite{samformat} lists i.a. sequential matches, mismatches, insertions and deletions and therefore represents the alignment of the corresponding segment and its reference sequence. \\
A BAM file is the binary representation of a SAM file. The main differences are the usage of a 4-bit encoding for the sequences, 3-bit for CIGAR Symbols and a 0-based instead of 1-based coordinate system for the position. \\
Furthermore, a BAM file is per default \textit{bgzf} compressed. This is a compression method that utilizes \textit{gzip} \cite{gzip} and the \textit{deflate} algorithm \cite{deflate} by Phil Katz to compress large files into blocks of less than 64KB size (compressed and uncompressed). These blocks are the concatenated allowing fast random access using index files but also compatibility with gzip as gzip allows this combination of multiple compressed files to one file. \\
Like gzip, bgzf supports compression levels ranging from 1 (fastest but worst) to 9 (slowest but best) which are basically the compression levels used for the underlying gzip compression.

\subsection{SAMtools}
SAMtools \cite{12ySamtools} is a collection of tools to work on alignment data. It relies on the co-developed HTSlib \cite{bonfield_htslib_2021} for reading and writing information files, e.g. BAM files. SAMtools offers functionality for different operations on alignment data, such as format conversion, statistics, variant calling and many more, including the sorting, which is the focus here.

\section{Prerequisites}

\subsection{SAMtools}
SAMtools~\cite{12ySamtools} is a collection of tools to work on alignment data, such as aligned DNA-Reads. It relies on the co-developed HTSlib~\cite{bonfield_htslib_2021} for reading and writing information files, namely SAM, BAM, and CRAM files. SAMtools offers functionality for different operations on alignment data, such as format conversion, statistics, variant calling and many more, including sorting, which is the focus here. 

\subsection{Aligned DNA-Reads}
DNA-Reads are short (typically 250–800 nucleotides long~\cite{hu_next-generation_2021}) sequences of nucleotides, the fundamental building blocks of DNA. These nucleotides are denoted by their bases, adenine (\texttt{A}), guanine (\texttt{G}), cytosine (\texttt{C}), and thymine (\texttt{T}). A DNA-Read can consist of multiple contiguous sequences. 

Aligned DNA-Reads are DNA-Reads aligned to a reference sequence. The alignment may include insertions, deletions, mismatches, and skipped parts of the reference sequence. Additionally, a step called \textit{clipping} excludes parts of the sequenced fragment with low read qualities from the alignment to improve the alignment of the remaining high-quality sequence with the reference. Also, changes in the direction of the alignment on the reference are possible. In alignment files, changes in direction of the alignment lead to splitting up the DNA-Read into multiple sequences, one for each contiguous part of the DNA-Read aligned in the same direction.

\subsection{SAM and BAM files}
A SAM (Sequence Alignment Map) file as specified by Li et al.~\cite{samformat} is used to store the alignment of sequences against reference sequences. It consists of a header section and an alignment section. The header section contains meta information such as the format version or the sorting of the content and a dictionary of the reference sequences, whereas the alignment section contains aligned segments with alignment information and meta information such as the read quality. A segment is a continuous sequence or subsequence of a raw DNA-Read. Aligned DNA-Reads are eventually put into multiple records with different segments in the alignment section, as single BAM records can not store changes in directions of the alignment on the reference sequence.

The alignment information primarily includes the ID of the reference sequence to which the alignment is mapped, the position in the reference sequence where the alignment starts, and detail on the alignment of this position (match, mismatch, insertion, or deletion). 

A BAM (Binary Alignment Map) file is the binary representation of a SAM file. Compared to the SAM format, this format utilizes a 4-bit encoding for DNA sequences, a 3-bit encoding for CIGAR symbols, and adopts a 0-based coordinate system for positions. Furthermore, a BAM file is per default \textit{BGZF} compressed.

\subsection{The DEFLATE format and algorithm}
The DEFLATE format~\cite{deflate} is a format specifying data compressed with a combination of an LZ77 algorithm~\cite{ziv_universal_1977} and Huffman Coding~\cite{huffman_method_1952}. The compressed data consists of subsequent blocks. The blocks contain strings and pointers to previous strings, which contain the distance to and the length of the matching part. The bytes of the strings which are no duplicates of previous found strings and the pointers, which consist of distances and lengths, are compressed using Huffman coding. The trees representing the Huffman codes are Huffman coded as well. 
The DEFLATE algorithm is an algorithm capable of producing an output stream in the DEFLATE format given an input data stream.

\subsection{GZIP and zlib}
GZIP~\cite{gzip} is a container format for data compressed in the DEFLATE format. A GZIP file consists of a series of \textit{members} which consist of a header, a series of compressed blocks, and a footer. The header contains meta information on the compressed files (e.g., the file name and the modification time) and can also contain extra fields which in the BGZF file format are used to identify a member as part of a BGZF file and store the length of the compressed member. The compressed blocks contain the compressed data and the footer contains a checksum and the size of the uncompressed content of the blocks modulo $2^{32}$. 

The c library zlib is an implementation of the DEFLATE algorithm. It can produce output in GZIP, ZLIB, or raw DEFLATE format. The ZLIB format is a wrapper for DEFLATE similar to the GZIP format but with less header information and another checksum algorithm (Adler32 instead of crc32).


\subsection{BGZF Compression} \label{bgzf}
The Blocked GNU Zip Format (BGZF), is a lossless compression method proposed together with the BAM format. Widely used compression methods like GZIP compress a file from the beginning to the end in one piece. This has the advantage of allowing matching segments of the file to be located over a greater range. Thus, the compression method is able to reduce the file size more effectively, as repeated sequences can be identified throughout the entire file. However, to decompress such a compressed file, it also needs to be read from the beginning and, depending on the compression method, decompressed at least until the point of interest. 

Given that not all regions of large alignment data files are relevant for every analysis, random access is required for efficient analysis of specific data subsets. To archive this, BGZF utilizes GZIP~\cite{gzip} to compress large files into blocks of less than 64\,KB size (compressed and uncompressed). GZIP uses the DEFLATE algorithm~\cite{deflate} to compress these individual blocks, which it then subsequently concatenates. Thus, fast random access using index files is possible. In an index file, the position of a piece of information is stored as a 64\,bit integer. This index is divided into a 48\,bit block index and a 16\,bit offset in the respective block.

The BGZF format leverages compatibility with GZIP, enabling any standard GZIP decompression tool to handle BGZF-compressed files. This compatibility stems from BGZF exploiting GZIP's ability to combine multiple compressed blocks into a single file. Given that Gzip is highly prevalent as a compression technique, there are numerous compatible compression and decompression libraries for all platforms. Thus, employing the open-source GZIP internally simplifies the development of other legacy tools working with BGZF-compatible compression.  

Like GZIP, BGZF supports compression levels ranging from 1 (fastest but largest file size) to 9 (slowest but smallest file size) analogously to the compression levels used for the underlying GZIP compression. The compression levels affect the size of the compressed aligned DNA-Read files, as shown in Figure~\ref{fig:compSizes}.
\begin{figure}[htb]
        \import{figures/}{compSizes.pgf}
    \caption{Comparison of the size of BGZF compressed files on all compression levels, exemplified using a 10.4\,GiB unsorted BAM file. 
    Although no compression yields a file four times as large, the distinctions between compression levels are less substantial.}
    \label{fig:compSizes}
\end{figure}
The speed of the compression depends mainly on the compression level, the GZIP-implementation and the number of used threads (see Figure \ref{fig:compSpeed}).  
\begin{figure}[htb]
        \import{figures/}{compSpeed.pgf}
    \caption{Comparison of the output rate of HTSlib's \texttt{bgzip} which uses BGZF to compress A 10.4\,GiB unsorted BAM file. For reference, compression level 0 is plotted in the smaller inset plot.}
    \label{fig:compSpeed}
\end{figure}
To measure the compression speed, we measure the speed of HTSlib's \texttt{bgzip}, a tool to compress arbitrary files using BGZF. \sort uses the same methods as \texttt{bgzip} of HTSlib, the library utilized by SAMtools for compression and file operations. This still holds for compression level 0. For this compression level, \texttt{bgzip} as well as \sort do not compress, but directly written the output. Therefore, the compression speed of \texttt{bgzip} is relevant for sorting because it sets a lower boundary on writing the output of \sort, considering that compression is a part of creating BAM files.

\FloatBarrier
\newpage % Zwischen Chapter kann man sich ein Seitenumbruch gönnen.
\section{Related Work}

\subsection{Sorting Tools}
SAMtools is the most commonly used tool for manipulating SAM and BAM files, which store information on aligned DNA-Reads. With over 5.1~Million downloads (05.2014), it is the most downloaded package from the Bioconda~\cite{the_bioconda_team_bioconda_2018} channel, which is a common source for bioinformatic tools in the package and dependency manager Conda. However, there are some alternatives for sorting aligned DNA-Read files, such as sambamba~\cite{tarasov_sambamba_2015}, Picard~\cite{Picard2019toolkit} and NovoSort~\cite{noauthor_novosort_nodate}. Picard, written in Java, provides a simpler interface for sorting SAM and BAM files, but its functions are limited compared to sort, and it does not prioritize efficiency. NovoSort is a commercial program and, therefore, I could not test it. Sambamba, written in the D programming language~\cite{alexandrescu_d_2010}, aims to provide a subset of SAMtools functionality, including sort, but with greater efficiency through higher parallelization. Sambamba reached this goal during its implementation in 2012 as SAMtools. However, with SAMtools introducing parallelism in version 1.4 in 2017, this advantage diminished.\\

\subsection{Compression Methods and Ouput Files}
Compression using BGZF, which is part of the specification of BAM files utilizes GZIP. As a general purpose compression method, GZIP is not specialized in compression DNA data. Hence, a variety of compression algorithms have been developed specifically for DNA data~\cite{hosseini_survey_2016}. These algorithms aim to minimize the resulting file size, compression speed, memory usage during compression or decompression, or decompression speed, but typically focus primarily on minimizing the resulting file size. Compression algorithms are divided in \textit{reference-based} and \textit{reference-free} methods. \\

Reference-based algorithms utilize the reference sequence a DNA-Read is aligned too. By only storing differences between each DNA-Read sequence and the reference sequence, they archive better compression. NGC~\cite{popitsch_ngc_2013} is an example for this, which also splits the kind of data, namely the sequences, sequence names, read qualities and so on in different blocks and compresses them seperately. Other algorithms like Goby~\cite{campagne_compression_2013}, DeeZ~\cite{hach_deez_2014}, and CRAM~\cite{fritz_efficient_2011} are also reference-based, but keep the advantage of enabling random access through index files. CRAM (Compressed Reference-oriented Alignment Map) is also built into HTSlib, the library SAMtools utilizes for file operations, and the user can choose to use it as output for \sort. NGC, Goby and CRAM also offer lossy compression. Lossy compression mainly removes meta information or applies lossy compression to the read qualities of DNA-reads, as their exact value is not always needed, it does not depend on the alignment and is thus hard to compress lossless. Although reference compression methods for SAM files are still actively developed~\cite{banerjee_abridge_2022}, they lack interoperability, as the compression methods are not widely used. An exception to this is CRAM, which is supported by HTSlib. Although CRAM files offer several advantages over BAM files, BAM files remain more popular due to their wider support among software tools.\\

Reference-free compression methods like BGZF, which is used in BAM files per specification, offer the advantage of not needing the reference sequence for compression and decompression. In addition, reference-based compression methods often require the DNA-Reads to be sorted. As this is not necessary for reference-free compression methods, they are more flexible. However, the only commonly used reference-free compression method is the BAM format, which uses binary encoding for the bases a DNA sequence consists of and BGZF compression. \\

In summary, the BAM file format employing BGZF compression, which internally utilizes GZIP, is the most commonly used format for storing aligned DNA-Reads. Given its widespread use, optimizing the speed of generating BAM files using \sort is essential. However, the increasing adoption of the CRAM format might lessen the future significance of BAM file generation.


\newpage
\section{Analysis (Version 1.19.2)}

\subsection{Algorithm}
\subsection{Prerequisites}
The process of sorting alternates depending on some internal Constants and command-line-arguments: 
We focus on sorting by the ID of the reference a DNA-Read is aligned to, then the position where the alignment on the reference starts, and then the REVERSE flag which indicates, if the sequence is aligned forward or backward to the reference. This order is used by SAMtools per default, although other sorting criteria e.g., tags or the read name are possible.

The maximum amount of memory \sort utilizes for storing BAM records during the sorting process is calculated by the amount of memory the user specifies via the \texttt{-m} option multiplied by the (via \texttt{-@} option) assigned number of threads. We refer to the total amount as \texttt{max\_mem}. 

Users can specify the number of threads to use for \sort by the \texttt{-@} parameter. If it is set to 1, the operation is single threaded. If the user sets it to a number greater than one, \sort uses the specified number of threads in addition to the main thread.

\sort infers the in- and output formats from the input and output file names the user specifies. SAMtools \texttt{sort} passes the sorted aligned DNA-Read files it outputs as BAM file to standard output if no output file is specified.

The maximum number of temporary files is hard-coded as 64 in a constant named \texttt{MAX\_TMP\_FILES}. 

Temporary files utilized in the sorting process are compressed with compression level 1. The compression level of the result file defaults to the default compression level used by the library that implements the compression. Usually this is compression level 6. The user can change the compression level of the output file via the \texttt{-l} parameter and set it to a number between 0 (no compression) and 9 (highest and slowest compression).

\subsection{Sorting} \label{sorting}

SAMtools performs an external memory sort utilizing temporary files that are merged in the end.
The sorting starts by sequentially reading BAM records from the input BAM file, or stream using HTSlib for parallel decompression. Once BAM records, which contain alignment information for DNA-Reads, exceed the memory limit given by \texttt{max\_mem}, \sort splits these BAM records into as many in-memory vectors as threads are specified and afterward sorts them in parallel. For sorting, each thread used for sorting (on multithreading, each thread except the main thread) performs a radix sort.

Then, \sort performs a heap based merge. In the merge, \sort keeps a binary heap containing the smallest entry from each file, and in-memory vector that should be merged. Each of these entries is a BAM record with the lowest order from the respective file or in-memory vector, together with a reference to the file or in-memory vector. The merge works by outputting the entry of the heap with the lowest order, inserting the next entry from the file or in-memory vector this entry came from into the heap, and adjusting the heap using the \textit{sift-down} algorithm~\cite{bojesen_performance_1999}. 

In the merge, \sort writes all the sorted in-memory vectors of BAM records to a single sorted temporary BAM file. In addition, \sort includes some of the previously created temporary files into the merge: The algorithm distinguishes between \textit{small files} and \textit{big files}. Small files are files generated by merging the sorted in-memory vectors of BAM records resulting from sorting them in parallel. If the number of small files is greater than half of the maximum allowed number of temporary files, \sort includes all small files it generated before into the merge (and deletes them afterward). The result of a merge of in-memory vectors of BAM records and temporary files is a big file. If the total number of files (small files and big files) exceeds the limit for temporary files, all previously generated temporary files including big files are included in the merge and deleted afterward. The resulting file is also counted as a big file, despite possibly being much larger than other big files generated by merging only small files. However, as the first merging of big files occurs at the 1120th temporary file\footnote{\label{limitReaching}This number is the result of adding $33 \cdot 33$ temporary files already merged into big files to $31$ small files. Here we have to square $33$, as $32$ small files can exist, and the $33$rd file is the big file which the result of a merge, but not counted among the small files. If $32$ big files exist, there is still space for $32$ small files, and they are merged to a $33$rd big file, leaving only space for $31$ small files in the next merging process.}, this is only relevant for combinations of enormous files and little memory. 

In general, temporary files can be put into three categories: small files being at most as big as the sorted in-memory vectors of BAM records together, big files being at most as big as half of the maximum number of allowed temporary files times the maximum size for small files and one big file growing depending on the ratio of allocated memory to the size of the input file possibly to much larger size than the other big files.

After the merge, the algorithm repeats the previous steps from reading the input to merging, until \sort reaches the end of the input file or stream. As a last step, \sort sorts the remaining in-memory BAM records and merges them together with all temporary files into the output file.
The sorting process flow is represented by the flowchart in Figure \ref{fig:flow}.

\begin{figure}[ht]
    \begin{adjustbox}{width=\linewidth}
        \tikzstyle{startstop} = [rectangle, rounded corners=5mm, thick,
minimum width=3cm, 
minimum height=1cm,
align=center, 
draw=darkgrey]

\tikzstyle{io} = [trapezium, rounded corners=0.5mm,thick,
trapezium stretches=true, % A later addition
trapezium left angle=70, 
trapezium right angle=110, 
minimum width=3cm, 
minimum height=1cm, 
align=center, 
draw=darkgrey]

\tikzstyle{process} = [rectangle, rounded corners=0.5mm,thick,
minimum width=3cm, 
minimum height=1cm, 
align=center, 
text width=3cm, 
draw=darkgrey]

\tikzstyle{decision} = [diamond, rounded corners=0.5mm, thick,
minimum width=3.2cm, 
minimum height=2.8cm, 
align=center, 
text width=1.9cm,
inner sep=2,
draw=darkgrey]
\tikzstyle{arrow} = [thick,->,>=latex,color=darkgrey]


\begin{tikzpicture}[node distance=2cm]
\tikzstyle{every node}=[font=\footnotesize]

\node (start) [startstop] {Start: \\ \#Files = 0 \\ \#BigFiles = 0\\ \texttt{MAX} := \texttt{MAX\_TMP\_FILES} \\Keep list of current \\temporary files\\};
\node (in1) [io, below of=start, yshift=-0.5cm] {Read BAM records \\ until memory full \\ or EOF};
\node (iseof) [decision, below of=in1, yshift=-0.5cm] {reached EOF?};
\node (pro1) [process, left of=iseof, xshift=-2cm] {Split into Threads blocks and sort them in parallel};
\node (dec1) [decision, below of=pro1, yshift=-1cm] {\vspace{-0.5cm}\\$\#\text{Files} - \#\text{BigFiles} >= \texttt{MAX} / 2$?\vspace{-0.5cm}};
\node (dec2) [decision, right of=dec1, xshift=2cm] {$\#\text{Files} >= \texttt{MAX} $?};
\node (consb) [process, below of=dec1, yshift=-0.5cm] {consolidate\_from := \#BigFiles};
\node (consf) [process, below of=dec2, yshift=-0.5cm] {consolidate\_from := \#Files};
\node (cons0) [process, below of=dec2, right of=dec2, yshift=-0.5cm, xshift=2cm] {consolidate\_from := 0};
\node (merge) [process, below of=consf, text width=13cm] {merge stored files from consolidate\_from to (\#Files - consolidate\_from) and all in
memory files into a file at position \#Files};
\node (consNotNull)[label={[xshift=-0.3mm, text width=2.3cm]center:consolidate\_from\\ \hspace{0.3cm}$>= \#\text{Files}$?}, decision, below of=merge, yshift=-0.5cm] {\phantom{consolidate\_from $>= \#\text{Files}$?}};
\node (updateFs) [process, left of=consNotNull, xshift=-2cm] {Remove merged files,\\ \#Files := \\ consolidate\_from \\ \#BigFiles := consolidate\_from + 1};
\node (addFile) [process, below of=updateFs, yshift=-0.5cm] {\#Files++};
\node(eof) [process, right of=iseof, xshift=2cm] {Sort remaining records in memory using 1 or all Threads depending on amount of records};
\node(end) [startstop, above of=eof, yshift=0.5cm] {Merge all \\ normal, big and \\in-memory files \\ and write final \\output};

\draw [arrow] (start) -- (in1);
\draw [arrow] (in1) -- (iseof);
\draw [arrow] (pro1) -- (dec1);
\draw [arrow] (dec1) -- node[anchor=south, color=black] {no} (dec2);
\draw [arrow] (dec1) -- node[anchor=east, color=black] {yes} (consb);
\draw [arrow] (dec2) -- node[anchor=east, color=black] {yes} (consf);
\draw [color=darkgrey,thick,->,>=latex,rounded corners=2pt] (dec2) -- ++ (3,0) -|  node[pos=0.93,left, color=black] {yes}  (cons0);
\draw [arrow] (cons0) -- (merge);
\draw [arrow] (consf) -- (merge);
\draw [arrow] (consb) -- (merge);
\draw [arrow] (merge) -- (consNotNull);
\draw [arrow] (consNotNull) -- node(yes)[anchor=south, color=black] {yes} (updateFs);
\draw [arrow] (updateFs) -- (addFile);
\draw [color=darkgrey,thick,->,>=latex,rounded corners=2pt] (consNotNull) -- ++ (0,-2) |-  node[pos=0.86,above, color=black] {no}  (addFile);
\draw [color=darkgrey,thick,->,>=latex,rounded corners=2pt] (addFile) -- ++ (-3,0) |-  (in1);
\draw [arrow] (iseof) -- node[anchor=south, color=black] {no} (pro1);
\draw [arrow] (iseof) -- node[anchor=south, color=black] {yes} (eof);
\draw [arrow] (eof) -- (end);

\end{tikzpicture}
    \end{adjustbox}
    \caption{Flow chart showing the current process of sorting, especially the choosing of files to be merged. The list of files is a 0-indexed list of their names. In the beginning it is empty, after BAM records are read the second time, there is a single record at position 0 and \#Files is 1.}
    \label{fig:flow}
\end{figure}
\FloatBarrier

\subsection{Time Allocation}
Understanding the resource utilization and the time allocation of the different parts of the sorting process is crucial to be able to optimize its computation time. However, the process has different points of constraint on different machines, as we will see in the following. \\
In general, high time consumption of the SAMtools \texttt{sort} method can be traced to three main blocks.

\subsubsection{Compression} is a part of writing BAM files, as per default compression is applied to all BAM files and even part of the specification. Although compression of BAM files is beneficial in the long term in order to reduce storing costs and transfer speed, it comes with a significant resource overhead. \\
Performing SAMtools \texttt{sort} on a laptop, trough various settings compression and decompression together account for around 95\% of the CPU time. Approximately 80\% are solely required by the \textit{deflate} method that is responsible for the compression.

\subsubsection{IO} can also be a constraint of the sorting process. As the internal mechanisms of SAMtools usually work very fast and are highly parallel, but have to process huge amounts of data, input and output devices can also limit the computation speed.

\subsubsection{Temporary Files} are necessary for SAMtools sort to work as a stream while processing more data than can be held in memory. Unfortunately, writing temporary files is time-consuming. Thus, the amount of temporary files should be minimized. More specific, a BAM record should be written as infrequently as possible. On the other hand, limitations of the Operating System have to be taken into consideration.

\subsection{Compression}
HTSlib is the tool used by SAMtools to perform all file operations. On its README, it claims its only dependency to be \textit{zlib}. zlib is a library used for compression utilizing the DEFLATE algorithm. 

\subsection{IO}


\subsection{Temporary Files}
SAMtools \texttt{sort} has, as mentioned above, a hard coded limit for temporary files. Moreover, this limit is reached very late because of multi level merging. \\
To understand how many temporary files are written, we have to look into the algorithm for merging. The first variable influencing the generation of temporary files is the memory limit. Defaulting to 768MiB, it gets multiplied by the number of threads. The result is the limit up to which BAM records are read in one block. This is also a good approximation for the size of a small temporary file before compression. At least one MiB per thread is enforced to prevent the creation of a huge amount of temporary files. Intuitively, one would think, that the sorting gets faster the more memory can be used. Figure \ref{fig:maxMems}
\begin{figure}
        \import{figures/}{maxMems.pgf}
    \caption{Execution time of SAMtools \texttt{sort} on a 2.4GB BAM file using default parameters except \texttt{-m} for memory limitation setting. }
    \label{fig:maxMems}
\end{figure}
\FloatBarrier
\newpage
\section{Temporary Files} \label{tempfiles}

Temporary files act as buffers for \sort, allowing it to process large datasets that exceed available memory while supporting streams as input and output options. However, writing temporary files is time-consuming, as \sort compresses each temporary file, writes it to disk, and then later decompresses and reads it again in order to merge the temporary files. 

When looking at the time between reading and decompressing the input and writing and compressing the output, Decompressing and compressing temporary files during the sorting process are more time-consuming than the actual sorting of BAM records, which store the alignment information of DNA-Reads in a BAM file. 

Furthermore, most operating systems limit the number of files that a process is allowed to keep open simultaneously. To address this constraint, \sort employs a merging strategy, reducing the number of open files in the final merge that creates the output files. However, in every merge of temporary files, BAM records, that have been compressed and written to a temporary file before are compressed and written again. This introduces overhead, as the computation time for compressing the content of the merged temporary files is added to the overall computation time. Thus, \sort should write as few temporary files as possible to reduce the frequency of each BAM record being compressed and written to a temporary file. In addition, a dynamic merging strategy that adapts to the limitations of the operating system can reduce the number of merges needed for a given number of temporary files, leading to performance improvements.

% 169GiB 0:44:51 [64.6MiB/s]
\subsection{Analysis}
\sort creates a total of 40 temporary files when sorting a 216\,GB unsorted BAM file utilizing 16 threads and a total of 32\,GiB memory. Since the main objective for temporary files is processing speed, not disk space, \sort compresses them with compression level 1 archiving maximal troughput compared to higher compression levels but reducing IO overhead from large file sizes. During writing, nearly all temporary files consume an average 29.75 seconds from opening the file to closing it. In this time, \sort merges the 16 (one per thread) sorted vectors of BAM records in memory and writtes them to the file. However, the 33rd file takes 945.38 seconds. That is 31.7 times the amount of time needed for the temporary files before.  

\sort performs merges of temporary files if a certain number of temporary files is reached. This is to limit the total number of temporary files needed for the final merge. In previous versions of SAMtools where this behavior did not exist, opening to many files at the same time in the final merge caused the program to crash.
To understand how many temporary files are written and when they are merged, one has to look into the algorithm for merging. 

 \sort enforces a predefined maximum number of temporary files that can be created during the sorting process. Until reaching half of this limit, it writes all blocks that are results of sorting the amount of BAM records fitting into memory at once to a single temporary file. If the limit is reached, the next temporary file is a merge of all small temporary files written before together with the next block of sorted BAM records in memory. In summary, on writing every 33rd file, \sort performs a merge of small temporary files. This explains the increase in time at writing the 33rd temporary file from the example above: \sort reads every temporary file written before again, merges them and writes their content a second time. As the \sort merges temporary files on half of the limit and generates a single file at every merge, the limit is reached later than the square of half the limit. If the limit is reached and 33 big files exist, \sort merges them again together with all small files and the records currently in memory. For details, refer to section \ref{sorting}. \\
The amount of merges depends on the number of temporary files needed in total. This is mainly determined by the amount of the memory the user gives to \sort. The user can set this limit using the "\texttt{-m}" parameter. Defaulting to 768\,MiB, it gets multiplied by the number of threads. The result is the limit up to which \sort reads BAM records in one block. This is also a good approximation for the size of a small temporary file before compression. At least one MiB per thread is enforced to prevent the creation of a huge amount of temporary files. One might instinctively believe that sorting becomes faster as more memory is utilized. Figure \ref{fig:maxMems} illustrates that this is generally the case, although not in a linear proportion.
\begin{figure}
        \import{figures/}{maxMems.pgf}
    \caption{Execution time of \sort on a 2.4\,GB BAM file using default parameters except \texttt{-m} for memory limitation setting. }
    \label{fig:maxMems}
\end{figure}
Moreover, between 400\,MiB and 12800\,MiB memory allocation the execution time does not decrease - despite \sort using up to 32 times more memory. To investigate further, one can take a look at the amount of temporary files produced. The input file expands to just a little larger than the second-highest memory limitation in Figure \ref{fig:maxMems}. Therefore, at the highest setting 25600\,MiB which equals to 25\,GiB, \sort produces no temporary file. At the next highest settings, it produces 1, 2, 4, ... temporary files, as the \texttt{max\_mem} parameter halves to every next highest value. Looking at the amount of temporary files generated, it is also possible to approximate the size of the BAM file in memory. At 400\,MiB, \sort generates 32 temporary files as expected. At 200\,MiB, it generates 65 temporary files. This indicates, that after having processed 12800\,MiB of data, 200\,MiB are not enough to keep the remaining data in memory until the final merge into the output file, but 400\,MiB are. For this reason, the size of the BAM file must increase to between 13000\,MiB and 13200\,MiB in memory. \\
Now it becomes obvious why there are no speed improvements between 400\,MiB and 12800\,MiB. In between those settings, \sort writes exactly the same records to the disk, in exactly the same order. The only difference is the number of files they are split into. \\
This changes at 200\,MiB \texttt{max\_mem}. The total of 65 produced temporary files means, that \sort has to perform a single merge and generate a single big file before the final merge. This comes with additional time consumption because \sort reads the content of the first 32 files from disk, decompresses, merges, compresses  and writes them to disk a second time. \\
At 100\,MiB \sort generates 3 big files, at 50\,MiB 7 and at 25\,MiB 15. This is also reflected in the total amount of bytes written. In the parameter settings that produce temporary files but not enough of them to be merged to big files \sort writes a total of 2.4\,GiB in temporary files. This number goes up to 3.7\,GiB, 4.3\,Gi, 4.6\,GiB and 4.8 GiB for 200\,MiB, 100\,MiB, 50\,MiB and 25\,MiB. Here, the increase in total written bytes for temporary files is not proportional to the amount of merges, as the size of the merged files shrinks with lowering the \texttt{max\_mem} parameter. In Addition, the proportional influence on the total time spend before merging the final result lowers with the number of performed merges: While writing the first big temporary file costs approximately as much as writing all temporary files before, writing the second one costs only a third of all file writing before, the next one 1/5 then 1/7 and so on.\\
Obviously, the measurements above are unrealistic, as nowadays even Laptops have more memory installed. At the same time, BAM files are usually way bigger than the used sample, which I sampled by randomly taking 1\% of BAM records from a real world BAM file. To get an impression of the impacts of increasing the file size, one can look at the changes that come with the size increase. \\
Both compression and decompression work in $\mathcal{O}(n)$, ensured by the blockwise compression. The sorting method used is a radix sort, which is also in $\mathcal{O}(n)$. For merging, a heap based approach is chosen, which works in $\mathcal{O}(n \log(k))$. Here, $k$ is the number of sorted list to be merged. Thus, in theory, keeping the same ratio of input size and available memory should produce the same amount of temporary files. Together with all other operations being in linear time, results on little files with little memory should transfer proportional to big files and more memory. This is confirmed by the experiment shown in Figure \ref{fig:memScaling}.\\
\begin{figure}
        \import{figures/}{memScaling.pgf}
    \caption{Execution time of \sort on different input sizes. Keeping the ratio from input size to \texttt{max\_mem} constant, the execution time grows linear with increasing both parameters.}
    \label{fig:memScaling}
\end{figure}
However, changing only one of these parameters has different effects. Using SAMtools for example locally installed on a laptop to sort a larger BAM file can produce many temporary files if memory is limited. If e.g. 8\,GB are available for \sort, it cannot process files bigger than 50\,GB without merging temporary files. This behavior gets worse if the ratio of the input file to the \texttt{max\_mem} setting grows further. An important point that should not be exceeded is reaching 1120 temporary files. At this point, \sort merges all "big files" into one single file. This means \sort writes every single BAM record it processed before to disk once more. This occurs approximately at sorting a 1700\,GiB file using 8\,GiB of memory, which is an unlikely use case. \\
In conclusion, writing larger amounts of temporary files leads to merging of temporary files, which is time-consuming. This is mainly affected by the ratio of the size of the input file to the amount of available memory.

\subsection{Recommendation}

Since the most time-consuming part of sorting is compressing and writing, the frequency of writing a single BAM record should be minimized. Therefore, \sort should perform as few merges as possible. \\
To archive this without changing any source code, the user can only change the "\texttt{-m}" parameter for memory limitation setting. The more memory the user gives to the process, the less likely \sort needs to merge temporary files. Therefore, the user should set this limitation as high as possible. However, as the memory limitation the user sets via "\texttt{-m}" is an upper bound only for storing BAM records in memory, SAMtools will most likely exceed it. Thus, the user should not set "\texttt{-m}" to the whole available amount of memory divided by the number of used threads, but keep some memory for SAMtools internal resource allocation. \\
Especially on laptops or for working on large files, the computing device provides not enough physical memory to avoid merging. Because of this, I recommend enlarging the limit for open temporary files. At the moment, it is set to 64 while modern computers are able to keep much more files open without noticeable performance losses. \\
On Unix systems, there exist two kinds of limits for the number of open files. The operating system differentiates between \textit{soft limits} and \textit{hard limits}.
A soft limit is a limit set by the user. If a process reaches the soft limit, the operating system kills it. On most modern systems, the soft limit is set to 1024 by default. \\
The hard limit is the limit up to which the user can increase the soft limit. Its size differs from system to system, but is typically much larger than the hard limit (e.g. 262144 on the computer I used for most of the experiments I present in this work). The hard limit can not be increased. \\
On a Unix operating system, a program can obtain its soft limit using the \texttt{getrlimit} \cite{noauthor_getrlimit2_nodate} system call. Knowing that \sort only opens all the files to merge, an output file (or standard output), possibly an index file and has standard input and standard error open, \texttt{sort} should recognize how many files can be opened and set the limit accordingly. Then, the user can also increase the soft limit, making merging of temporary files obsolete for realistic use cases. For compatibility reasons, if the system call fails, the limit can be kept.
%% but increased???
This is e.g. on Windows machines necessary due to Windows not having a limit for open file handles and thus not supporting \texttt{getrlimit}. \\
Notice, that the necessary file size until the limit is reached grows quadratic to the maximum amount of temporary files \sort allows. On the other hand, the file size up to which \sort does not perform a merge of temporary files grows only half as fast as the maximum number of temporary files.

\subsection{Evaluation}
Increasing the number of allowed temporary files to 1019 ($=1024-5$) while keeping everything else the same leads to a 15.5-fold increase in the potential file size before triggering a merge. \sort then performs the first merge of temporary files at the 513th file instead of at the 33rd. Having a limited amount of 8\,GiB of memory, the change to 1019 allowed temporary files raises the tipping point, after which the first merge of temporary files occurs, from around 50\,GiB input size to around 775\,GiB input size.
\begin{figure}
        \import{figures/}{speedupMems.pgf}
    \caption{Speedup after setting the limit for temporary files to 1019. Calculated by dividing the values from Figure \ref{fig:maxMems} by the values of the increased temporary file limit. All other parameters are the same as in Figure \ref{fig:maxMems}.}
    \label{fig:memSpeedup}
\end{figure}
Figure \ref{fig:memSpeedup} shows, that in the example I presented before, a noticeable speedup occurs only at the lower memory settings but not at the lowest. One can understand this by referring to Figure \ref{fig:writes}.
\begin{figure}
        \import{figures/}{writes.pgf}
    \caption{
    The y-axis shows the amount of compress and write operations for blocks in size of the available memory. This means, if one small temporary file is produced, it counts as one write operation. If 32 files are merged together with one in-memory block, it counts as 33 Operations. Vertical gray lines mark numbers of temporary files produced in the example in Figure \ref{fig:maxMems} and \ref{fig:memSpeedup}. (E.g., 527 at 25\,MiB \texttt{max\_mem}.)
    }
    \label{fig:writes}
\end{figure}
Figure \ref{fig:writes} shows the number of times, a block of BAM records in size of the available \texttt{max\_mem} is compressed and written into a temporary file. Having a limit of 64 temporary files, files are merged relatively often compared to having a limit of 1019 temporary files. Therefore, the graph for the smaller limit has much more steps. On the other hand, changing the limit for temporary files also means, that if temporary files are merged, much more of them are merged at once. The gray, vertical lines mark the number of temporary files \sort creates at the different settings in the example in Figure \ref{fig:maxMems} and \ref{fig:memSpeedup}. If the amount of temporary files is below 33 files, which is true for the seven highest memory settings, the total amount of decompression and write operations (in memory-sized blocks) is equal for both limits. After this, the larger limit gains advantage until the 509th temporary file. Writing the 510th temporary file with a limit of 1019 temporary files means, that \sort rewrites the content of all 509 files it generated before again. For some temporary files, the smaller limit even needs less writes in total. This happens, because after the first merge at the greater limit, the content of every file \sort generated before is written twice, except the content of the in-memory block of BAM records that is taken into the merge that results in file number 510. However, considering the smaller limit, at the same number of files nearly all content of temporary files is written twice as well. Here, exceptions are again the BAM records being an exclusive part of a big file, as they are taken into a merge together with small files but never written into a small file on their own. As this happens much more frequent at the smaller limit, the smaller limit needs less compress and write operations every time the greater limit performs a merge. This changes again at the next merge at the smaller limit. \\
In summary, increasing the limit for temporary files results in a noticeable speedup if it prevents merges. However, if a merge with a greater limit is performed, the impact on the execution time is stronger than with a smaller limit. If the limit is calculated from the soft limit defined by the operating system, it is maximized and the user can increase it if necessary. 

\subsection{Future Work}
Even after setting the limit for temporary files to 1019, \sort merges temporary files the first time at writing the 510th temporary file. Due to merging, \sort reaches the real limit of 1019 files after writing more than 260000 files. While with the proposed changes the user can prevent merges by changing the soft limit, knowing about this option is unlikely for an average user. Thus, \sort should use as much of the limit as early as possible to keep the number of merges as low as possible. This can be archived by writing small temporary files not only up to half of the limit for temporary files, but to the limit minus the current amount of big files before merging them. For the first merges, this change would lead to reducing the number of merges by half.
\newpage
\section{Compression} 
Compression is a part of writing BAM files. BAM files are per specification BGZF compressed. BGZF internally uses gzip for compressing small blocks of a file, which BGZF then concatenates. Although compression of BAM files is beneficial in the long term in order to reduce storing costs and increase transfer speed, it comes with a significant resource overhead. 

\subsection{Analysis}
Running on 16 threads, with a total of 32\,GiB of memory, \sort takes 71 minutes and 57 seconds to sort a 215\,GB BAM file. However, it uses only 2 minutes and 35 seconds, which are 3.6\% of the total time, for sorting (merging not included). The profiler VTune \cite{noauthor_fix_nodate} reveals that much of the remaining time is dedicated to compression. \\ 
Compression and decompression together account for around 95\% of the CPU time when performing SAMtools \texttt{sort} on a laptop, trough various settings. Solely the \textit{deflate} method that is used for the compression and a part of zlib, requires approximately 80\% of the CPU time. \\
SAMtools has outsourced all file operations to HTSlib. HTSlib depends on zlib for compression and decompression. Compression is done in blocks using the DEFLATE algorithm. Thus, compression and decompression are parallelized: Every time a block is to be compressed or decompressed, HTSlib gives it to a thread pool of workers that compress blocks in parallel.


\subsection{Compression Levels}
Compression Levels are the gzip way of trading computation time against space requirements. The user can set them using the "\texttt{-l}" parameter in SAMtools sort. For other SAMtools commands where the "\texttt{-l}" parameter does not exist, the user can still change the compression level of the output via adding \texttt{--output-fmt-option level=1} to the arguments of the command (Put the desired compression level between 0 and 9 instead of \texttt{1}). \\
In general, SAMtools uses two different compression levels. For writing output files, it uses the default compression level of the zlib implementation (compression level 6 for zlib). This can be changed as explained above. For temporary files, SAMtools uses compression level 1. In current SAMtools versions, this can not be changed without changing the source code.
Differences between the compression levels are shown in Figures \ref{fig:compSizes}, \ref{fig:compSpeed}, \ref{fig:bgzfComps} and \ref{fig:bgzfspeed}.

\subsection{Alternative zlib Implementations}
Being build into the Linux kernel, zlib is seen as the de facto standard of file compressing using the DEFLATE algorithm. The first version of zlib was published in 1995 to be used in the PNG handling library \textit{libpng}. Although still maintained, other libraries have been created that surpass zlib in both compression speed and ratio.\\
For Example, \textit{libdeflate} \cite{biggers_ebiggerslibdeflate_2024} offers faster compression while archiving a better compression ratio at the same time. Libdeflate archives this through various improvements such as using word access instead of byte access in input reading and match copying, which are parts of the DEFLATE algorithm. Furthermore, it uses a speed-up Huffman decoding process, loads the whole block into a buffer before compressing and utilizes BMI2 instructions on x86\_64 machines if they support them. \\ %% fallen ziemlich vom himmel die sachen. Ich wollte eigentlich nur einen kkeinen Eindruck verschaffen, wieso das schneller ist, aber wahrscheinlich muss ich da entweder tiefer rein, oder das ganz weglassen, oder?
Support for libdeflate is already built into SAMtools. Moreover, the developers recommend using libdeflate instead of zlib. If HTSlib's configure script finds libdeflate libraries, HTSlib uses them automatically instead of zlib. To decide manually between using zlib and libdeflate, the user can run the HTSlib \texttt{configure} script with the \texttt{--with-libdeflate} resp. \texttt{--without-libdeflate} option. If it is configured to use libdeflate, HTSlib uses the libdeflate API for compression and decompression. As libdeflate supports 12 compression levels instead of 9 compression levels supported by zlib, SAMtools maps the compression levels as shown in Table \ref{tab:levelMapping}. \\
\begin{table}[]
    \centering
    \begin{tabular}{l|>{\hspace{0.1em}} c >{\hspace{0.1em}} c >{\hspace{0.1em}} c >{\hspace{0.1em}} c >{\hspace{0.1em}} c >{\hspace{0.1em}} c >{\hspace{0.1em}}c >{\hspace{0.1em}} c >{\hspace{0.1em}} c}
         zlib & \hspace{0.1em} 1 & 2 & 3 & 4 & 5 & \textbf{6} & 7 & 8 & 9 \\
         libdeflate \hspace{0.1em} & \hspace{0.1em} 1 & 2 & 3 & 5 & 6 & \textbf{7} & 8 & 10 & 12 \\
    \end{tabular} \vspace{1em}
    \caption{Mapping between zlib compression levels and libdeflate compression levels in HTSlib. The default level is marked \textbf{bold}.}
    \label{tab:levelMapping}
\end{table}

In addition, the user can choose to use other zlib implementations by using \texttt{LD\_PRELOAD} \cite{myers_intercepting_nodate-1}. \texttt{LD\_PRELOAD} is an environment variable telling the loader to load shared libraries. A shared library is a code object which is not part of a single program but can be used by multiple programs. Methods and symbols from the shared library are connected to other programs by the linker. If two different definitions for symbols exist, the linker prefers the one from a shared library in \texttt{LD\_PRELOAD} over the one from a shared library that is not in \texttt{LD\_PRELOAD}. \\
For example, HTSlib uses the \texttt{deflate} method of \texttt{libz.so}. However, the user can compile e.g. \textit{zlib-ng} \cite{noauthor_zlib-ngzlib-ng_2024}, which is API compatible to zlib, to a shared object. Then he can specify the path to the compiled shared object in \texttt{LD\_PRELOAD}. As a result, every time HTSlib calls zlib methods which are also implemented in zlib-ng, it uses the implementations in zlib-ng of the implementations in zlib. \\
However, this approach is only possible, if the replacement implementation supports the zlib API. Other libraries, which also produce gzip compatible output but use a different API, can not be used for this approach. 

\subsection{7BGZF}
\textit{7BGZF} \cite{yamada_7bgzf_2020} is a tool developed by Taiju Yamada for testing different GZIP compatible compression libraries. It works by overwriting \texttt{bgzf\_compress}. This is the method HTSlib uses for compressing. When running SAMtools with 7BGZF, 7BGZF chooses the library HTSlib uses for compression based on the \texttt{BGZF\_METHOD} environment variable. The user can set the \texttt{BGZF\_METHOD} environment variable to a compression library concatenated with a compression level before running a SAMtools command. This approach has the advantage, that it simplifies testing different compression libraries. 
To test 9 libraries, the user only has to do one installation, as 7BGZF's only dependency is \texttt{libc}. Moreover, not all libraries are drop-in for zlib. For igzip for example, 7BGZF has to use a different API. Utilizing 7BGZF abstracts away from the zlib implementation's different APIs, simplifying the switch to another zlib implementation with just a single environment variable adjustment.
\\
On the other hand, using 7BGZF has some disadvantages to using \texttt{LD\_PRELOAD}. SAMtools has to link to a shared library rather to linking to the static library, as the method 7BGZF overwrites is a method of HTSlib. To archive this, the user has to change a single line in SAMtools' \texttt{config.mk.in} and change \texttt{@Hsource@HTSLIB = \$(HTSDIR)/libhts.a} to refer to \texttt{libhts.so} instead. \\
Moreover, 7BGZF does not distinguish between different compression levels, which are parameters of the overwritten \texttt{bgzf\_compress} method. Instead, 7BGZF receives the compression level to be used as part of the \texttt{BGZF\_METHOD} environment variable. Therefore, it applies the same compression level on every written BGZF compressed file. 
In the context of \sort, this means temporary files have the same compression level as output files, which \sort compresses with compression level 1 per default. This leads to more time consumption for compression if the output file should use a compression level which is not level 1. Therefore, using files that require \sort to produce temporary files distort comparisons with default zlib or libdeflate implementation. \\
Testing 7BGZF on sorting a BAM file small enough not to produce any temporary files, still gives hints on which libraries to use for faster sorting. For libdeflate and zlib, \sort achieved similar runtimes on average when using 7BGZF compared to plain HTSlib, with variances ranging from 3 to 5 percent. \\

7BGZF claims to support 9 different compression libraries. 
After using the compiling scripts 7BGZF provides, only seven of them are working. Compression by using 7zip and crypto++ fails and defaults to zlib compression. However, the following implementations still work and are tested here:\\
Next to zlib and libdeflate (see above) \textit{zlib-ng} \cite{noauthor_zlib-ngzlib-ng_2024} by Hans Kristian Rosbach is a merge of optimizations of a now archived zlib version by Intel \cite{noauthor_intelzlib_2024} and a fork by Cloudflare \cite{noauthor_cloudflarezlib_2024}. Both merged implementations can be found in old zlib implementation comparisons. The idea behind zlib-ng is to provide a version of zlib witch is more receptive for code changes. Also, due to less need of working for very old systems and compilers, many of zlib's workarounds are removed. Mark Adler, the maintainer of zlib also regularly contributes to zlib-ng.\\
Googles \textit{zopfli} \cite{noauthor_googlezopfli_2024} is an algorithm designed by Lode Vandevenne and Jyrki Alakuijala to enable the best possible deflate compatible compression by finding the best parameters for deflate. However, its implementation is very slow compared to other zlib implementations. \\
In contrast, \textit{igzip} \cite{tucker_isa-l_2017}, which is a part of the Intelligent Storage Acceleration Library (ISA-L) \cite{noauthor_intelisa-l_2024} by Intel, focuses on compression as fast as possible on cost of the compression ratio. \\
\textit{miniz} \cite{noauthor_richgel999miniz_nodate} by Rich Geldreich is another ground up zlib implementation being in a single source file. \\
The last working library is \textit{slz} \cite{tarreau_wtarreaulibslz_2024} by Willy Tarreau. It aims at reducing resource usage for web servers by using simpler encoding and matching, which are parts of the DEFLATE algorithm. This changes also come with performance improvements but lower compression ratio.\\

Most of these libraries offer the GZIP-typical compression levels of 0 up to 9 and level 6 as default. Exceptions are libdeflate with compression levels up to 12, igzip with compression levels 0 to 3 defaulting to 1 and slz providing only compression on level 1. For miniz and zopfli, the default compression level is 1. Default compression levels are important to know because of SAMtools using the zlib implementation's default compression level, if no compression level is set via "\texttt{-l}" or "\texttt{--output-fmt-option}". A comparison of features of the compression libraries tested in 7BGZF is shown in Table \ref{tab:libs}.

\begin{table}[]
  \renewcommand{\arraystretch}{1.2}%
    \centering
    \begin{tabularx}{\textwidth}{l|*{7}Y}
         Implementation \hspace{0.5em} & zlib & libdeflate & miniz & igzip & slz & zlib-ng & zopfli  \\
         \hline
         Levels & 1-9 & 1-12 & 1-9 & 1-3 & 1 & 1-9 & -\footnotemark \\
         Decompression & yes & yes & yes & yes & no & yes & no \\
         Drop-In\footnotemark & - & no & yes & no & no & yes & no
    \end{tabularx}
    \vspace{1em}
    \caption{Comparison of features of the compression libraries tested in 7BGZF.}
    \label{tab:libs}
\end{table}
\footnotetext[2]{zopfli does not use compression levels but can specify iterations of the algorithm. Here, the level I show in experiments is always used as iterations.}
\footnotetext{Drop-In does not mean, that the API matches to 100\% but that the most important symbols are implemented}

\subsection{7BGZF Results}

On a limited number of threads, specifically fewer than four, we can archive a speedup of up to 5 using BGZF. To work around 7BGZF's limitation regarding separate compression levels for output and temporary files, I tested it on a file small enough to avoid producing temporary files in \sort. The results can still be used to see which libraries have potential for replacing zlib in HTSlib. I primarily compare them to zlib and libdeflate, since HTSlib already supports these.\\
The best performance could be reached by igzip. Both tested compression levels compressed faster than all compression levels of all the other compression libraries. Using one or two threads, they archived a speedup of up to 5 compared to the default zlib compression. However, the compression rate of both tested compression levels turned out lower than libdeflate on compression level 1 (30\% respective 23\% of the original size for igzip on compression level 1 respective compression level 3 against 22\% of the original size for libdeflate on compression level one, see Figure \ref{fig:bgzfComps}). \\
\begin{figure}[!htb]
        \import{figures/}{compRatiosBGZF.pgf}
    \caption{Comparison of the compression ratio of different zlib implementations. Sizes are relative to the uncompressed file. \\
    igzip on compression level 1 together with slz and zlib-ng on compression level 1 produce files 50\% larger than zlib on default compression level. Miniz on compression level 1 and zlib on compression level 1 produce 20\% larger files than default zlib. All other settings resulted in differences of up to 10\% which are 2\% of the uncompressed file.}
    \label{fig:bgzfComps}
\end{figure}
\begin{figure}[!htb]
        \import{figures/}{bgzfSpeedup.pgf}
    \caption{Speedup of \sort using 7BGZF relative to the zlib compression with compression level 6 which is \sort's default compression method. In the legend, libraries are sorted after their speedup on a single thread. The compression level is noted in brackets. Sorting the 2.3\,GiB default compressed BAM file with 48GiB of memory did not produce a temporary file. \\
    Faster compression libraries and lower compression levels, such as igzip, slz, zlib-ng on compression level 1 and libdeflate on compression level 1 are 3.5 to 5 times faster than the default compression if up to two threads are used. Libdeflate and zlib-ng on compression level 6 archive a similar speedup as zlib on compression level 1.}
    \label{fig:bgzfspeed}
\end{figure}
While slz archives a speedup of 4.5 on one and two threads  and is therefore nearly as fast as igzip on level 3, it compressed the tested files to 30\% of its uncompressed size, which is more comparable to igzip on compression level 1. \\
Zlib-ng is comparable with libdeflate but offers a wider range. While zlib-ng on compression level 1 archives a speedup of 4 for one or two threads surpassing libdeflate on the same level, it also produces 33\% bigger files. However, zlib-ng on compression level 6 produces a 3\% smaller file than libdeflate on the same level, but archives a speedup of less than 2 making it noticeably slower than libdeflate on compression level 6. \\
Libdeflate had a speedup of 2.3 for up to 4 used threads, while producing a 0.5\% smaller file than the default zlib compression. This speedup is even larger than the speedup of 2.1 resulting from using zlib on compression level 1. On compression level 1 libdeflate archived a speedup between 3.5 and 4 for up to 4 threads while producing a 6.5\% larger file than the default zlib compression. \\
Reducing the compression level to 1, zlib reaches a speedup of approximately 2 for up to 8 threads, while producing a 20\% larger file than on the default compression level of 6. \\
Zopfli and miniz are excluded for clarity. Previous experiments showed that zopfli produced around 10\% smaller files than the default zlib compression, but took 70 to 90 times longer.   
Miniz provides for each level a marginally worse compression ratio than zlib and also has longer processing times, as previous experiments showed. \\

Increasing the number of threads gradually reduces the speedup achieved by using faster compression libraries. The speedup of 5 igzip provides on one or two threads diminishes to 4 on 4 threads, to 2.5 on 8 threads and finally to 1.5 on 16 threads, the highest tested number of threads. For all other compression libraries and compression levels, igzip at compression level 1 serves as an upper limit. The speedup of all other libraries stays close to their single-thread speedup until the number of threads, where the speedup of igzip on compression level 1 drops below their single-thread speedup. Then their speedup approaches the speedup of igzip on compression level 1. Therefore, on 16 threads, all alternative zlib implementations on their tested compression levels archive a speedup of approximately 1.5. \\
The speedup to the single-threaded execution time of the different compression libraries and compression levels increases slower for the settings archiving a higher single-thread speedup. The speedup of zlib on compression level 6 increases with every higher number of threads, reaching a speedup of 12 on 16 threads. In contrast, faster compression libraries such as igzip show a less significant increase in speedup. All tested compression libraries with the corresponding compression levels that archived a speedup as high or higher than libdeflate on compression level 1 compared to zlib on compression level 6 on a single thread reached their highest speedup against their single-thread performance on 8 cores with a speedup of less than 5. When raising the number of threads from 8 to 16, their speedup decreases slightly, as shown in Figure \ref{fig:bgzfSngleCoreSpeedup}.
\begin{figure}[!htb]
        \import{figures/}{singleCoreSpeedupBGZF.pgf}
    \caption{Speedup against single-threaded performance of 7BGZF compression libraries and compression levels. Sorting the 2.3\,GiB default compressed BAM file with 48GiB of memory did not produce a temporary file. \\
    The speedup of zlib on compression level 6 increases with every higher number of threads. In contrast, faster compression libraries such as igzip show a less significant increase in speedup. When raising the number of threads from 8 to 16, their speedup decreases slightly.}
    \label{fig:bgzfSngleCoreSpeedup}
\end{figure}
This suggests that beyond a certain number of threads, compression, which is fully parallelized, is no longer the limiting factor. This is supported by the observation that the relative time \sort spends on decompression, sorting and compression converges to a similar ratio for all compression libraries on the tested levels. On 16 threads, all compression libraries on the tested levels used 21\% to 31\% of their computation time for decompression, 6\% to 9\% for sorting and 60\% to 71\% for compression. Here, the faster compression methods use more of their computation time for decompression than slower methods like zlib on compression level 6. On a single thread, the faster libraries use much more, around 50\% of their time for decompression, while zlib on level 6 only uses 9\% of its computation time for decompression. On two threads, the percentage of the computation time the compression methods with a higher single-thread speedup spend on compression decreases further, but grows again with more used threads. In contrast, for the compression methods with less single-thread speedup, the relative computation time spend on compressing decreases gradually up to around 70\% on 16 cores. This is illustrated in Figure \ref{fig:relative7BGZF}.\\
\begin{figure}[t]
        \import{figures/}{relative7BGZFtimes.pgf}
    \caption{}
    \label{fig:relative7BGZF}
\end{figure}

Besides compression, decompression speed should be taken into account for the full picture. Here, differences between decompressing files compressed by 7BGZF compression libraries are in a range of 15\% around the average decompression time of compressed files for both decompression libraries.
\begin{figure}[t]
        \import{figures/}{decomp.pgf}
    \caption{Execution time of SAMtools \texttt{view} with \texttt{-u} flag on a single core, piped to \texttt{/dev/null}. The test file is a 104\,GB uncompressed BAM file, compressed by \sort using the 7BGZF settings shown on the x-axis. HTSlib was build with \texttt{--with-libdeflate} for decompression utilizing libdeflate respective \texttt{--without-libdeflate} for the decompression utilizing zlib.}
    \label{fig:decomp}
\end{figure}
Which compressed file is decompressed the fastest varies between the two decompression libraries. Nevertheless, there is a trend indicating that files with smaller sizes, thus higher compression rates, are decompressed faster, as shown in Figure \ref{fig:decomp}. 
Additionally, it is evident that libdeflate is significantly faster than zlib for decompression. This still holds for the uncompressed file ("uncomp" in Figure \ref{fig:decomp}). This is due to a checksum calculation. For each read block, HTSlib calculates a crc32 checksum. Looking at SAMtools \texttt{view} in a profiler, the implementation for the crc32 checksum in libdeflate turns out to be about 10 times faster. \\

In conclusion, igzip, slz as well as zlib-ng and libdeflate on compression level 1 are very fast and suitable for temporary files and files to be used soon. On compression level 6, zlib-ng and libdeflate are good default values and provide a trade-off between file size and computation time. Zopfli offers very good compression, but the huge increase in computation time makes using high compression levels of zlib-ng and libdeflate more preferable in most settings for \sort.

\subsection{Recommendation}
The most obvious way to increase compression speed is to increase the number of available threads. This can be done using the "\texttt{-@}" parameter. Threads are also used for parallel sorting of in memory blocks of read BAM records. This further speeds up the process. \\
For compression levels, it comes down to what the purpose of the sorted data is when deciding the level to be used. 
If the file should be archived, setting the level to 9 for maximal compression is an option. However, most of the time, sorting is a step in a larger pipeline, and the data is read and processed further soon. \\
To speed this up, I recommend to use compression level 0 ("\texttt{-l 0}" which is equal to "\texttt{-u}") if the data is not written directly to  disc or transferred over network with limited throughput. \\
If the data is written to disk directly or transferred via network, it comes down to the IO conditions which level to choose (see next section). In most cases, compression level 1 ("\texttt{-l 1}") is a good starting point.\\
When it comes to the zlib implementation to use, I recommend libdeflate. While not being the very fastest implementation, it is already supported by SAMtools. This means the stability is much higher as its usage is tested with SAMtools. Also, the faster crc32 implementation leads to performance improvements. At last, installing and using libdeflate is simpler than using \texttt{LD\_PRELOAD} or changing the environment for every user. After installation, the user can use SAMtools exactly as he has done before without libdeflate, but with better performance.

\subsection{Evaluation}
\begin{figure}[t]
        \import{figures/}{speedupComps.pgf}
    \caption{Speedup of \sort after changing compression parameters. Reference is \sort using HTSlib with default zlib compression. Libdeflate is used via the HTSlib integration, igzip via 7BGZF. The input file is a 23.6\,GB unsorted BAM file with default zlib compression. \sort had a total of 48\,GiB memory available. The output is piped to \texttt{/dev/null} to minimize IO impacts. Still, limitations in the writing of temporary files or in reading can distort the impacts of changing only compression properties. \sort compresses the 23.6\,GB compressed input file to 21,1\,GB using zlib (the reference), 20.5\,GB using libdeflate on level 6, 22,6\,GB using libdeflate on level 1 and 31.5\,GB using igzip on level 1 (in both settings). \\
    For one to four threads \sort uses, the speedup with using faster compression is up to 5. However, with more threads available, the speedup decreases to approximately 1.5 for all compression methods.}
    \label{fig:speedupCompression}
\end{figure}
\begin{figure}
        \import{figures/}{speedupFinalCompsSingleCore.pgf}
    \caption{Speedup of \sort after changing compression parameters. The reference is the respective single-core performance (strong scaling). The default zlib implementation benefits most from a higher number of threads. The faster the compression speed, the less benefits come from using a higher number of threads.}
    \label{fig:speedupCompression}
\end{figure}
Reducing the compression level and choosing a faster zlib implementation leads to a speedup of up to 5 for single core sorting. As the compression part in \sort is highly parallel, increasing the number of threads also speeds up the process. This lowers the impact of the compression on the total execution time. Therefore, the speedup lowers with increasing the number of threads to 1.5 for all tested compression settings, as shown in Figure \ref{fig:speedupCompression}. \\
In conclusion, to reduce computation time of the compression done by \sort, the user can lower the compression level or choose a different zlib implementation. Libdeflate emerged to provide higher compression with lower computation time in both compression and decompression. Other libraries like igzip offer faster compression than libdeflate, but are currently not supported by HTSlib. 7BGZF is a tool for testing 7 different zlib implementations for compression in HTSlib.


\subsection{Future Work}
Intel's igzip performs even better than libdeflate. Although this comes with the downside of larger files, implementing igzip support in HTSlib would enable even faster compression for temporary files. E.g. a mapping could be used mapping compression level 1 and 2 to compression level 1 and 3 of igzip and the higher levels to libdeflate levels. \\
For improving 7BGZF, a differentiation between output files and temporary files could be implemented.
\FloatBarrier
\newpage
\section{Input/Output} 
Input and Output can also be constraints of the sorting process. As the internal mechanisms of SAMtools usually work very fast and are highly parallel, but process huge amounts of data, input and output devices can also limit the computation speed. \\
In contrast to a program's behavior or used libraries, often the IO devices can hardly be changed. Nevertheless, in some cases there are possibilities to speed up the process.

\subsection{Compression}
Lowering the compression rate by using a faster zlib implementation or a lower compression level leads to faster compression. Setting the compression level to 0, the user can archive even faster output. \\
However, taking IO requirements into account, compressing files turns out to be faster than writing them uncompressed if the disk does not match the compression throughput. In this case, writing uncompressed files is slower than compressing them first. As compression is done blockwise with every block compressed individually, it scales well with increasing the amount of threads. Therefore, IO bottlenecks occur with higher probability if using a higher amount of threads. \\
On our test system, removing the output compression leads to a speedup of approximately 3.5 on a single thread. However, on 16 threads, running \sort with the default compression level 6 takes only 62\% of the time it takes with uncompressed output (see Figure \ref{fig:execIO}). \\
\begin{figure}[t]
        \import{figures/}{execIOBottleneck.pgf}
    \caption{Execution time of \sort with output written to disk. Writing uncompressed Data is faster for up to two threads. If \sort has more than two threads available, writing compressed BAM files is faster. Compression at compression level 1 is still faster than compression at compression level 6 (the default level).}
    \label{fig:execIO}
\end{figure}
The reason for this is the limited write speed of the disk, which is at 56.5\,MB/s for our test setup. With uncompressed output setting, \sort writes 2.22\,GB in 10.5\,s if piped to \texttt{/dev/null}. This results in a throughput of 211.7\,MB/s in the writing phase of \sort. Due to the disk's inability to maintain the pace, the CPUs have to pause and await the completion of IO operations. The time a CPU waits in idle for IO Operations is called IOWait time. It is below 2 seconds and nearly exactly the same for both compressed output from the example above, but increases with the number of threads used from 2 seconds to 6 seconds for the uncompressed output, as shown in Figure \ref{fig:iowait}.
\begin{figure}
        \import{figures/}{iowaitLevels.pgf}
    \caption{IOWait time of the experiment shown in Figure \ref{fig:execIO}. IOWait time is the time all processors spend in IDLE because they are waiting for IO operations to return. For compression with libdeflate using compression level 1 and 6, the IOWait time is on a similar level. After at more than one core, it is nearly constant. For uncompressed output, the IOWait time increases to up to 6 times the IOWait time at compression usage. }
    \label{fig:iowait}
\end{figure}



\subsection{Unix Pipelines}

\textit{Pipelines} are a way of forwarding the output of one program to the input of another program. In Unix-like operating systems, they connect the standard output of one program with the standard input of another program. In the Linux Kernel, this is implemented \cite{noauthor_linuxfspipec_nodate} by a ring buffer in the memory. The user usually expresses a pipe in the shell by writing a vertical bar "\texttt{|}" between the commands. For example, 
\begin{minted}{bash} 
ls | grep .bam 
\end{minted}
lists all files in the current directory using \texttt{ls} and then filters them using \texttt{grep} to display only files having ".bam" in their name. 

\subsection{Pipelining in SAMtools}
Pipelining enables all advantages of working with streams. For \sort, this has the following consequences: \sort can read BAM records as soon as a potential preceding command has produced them. Here, if the potential preceding command supports this, he can give uncompressed BAM data to \sort. This eliminates the need for decompression for \sort and for compression for the potential preceding command. It also eliminates writing and reading files to the disk in between the commands. Note that the process generating the input for SAMtools \texttt{sort} is most likely halted once the memory limit of \texttt{sort} is reached. This is due to the buffer offered by the pipeline being full and thus the write operation being blocked. \sort only empties the buffer again after a temporary files is written. Looking at the output of sort, there is also no need for compression and writing to disk if piped. \\
Pipelining can be used to chain multiple SAMtools commands without the need to write temporary files in between them, as all SAMtools commands are working on streams. This can result in a huge speedup, as compression and writing account for a significant portion of \sort's computation time. In addition, as the outputs are streams, the second command can start processing the output of the first one as soon as it begins to be generated, instead of having to wait for the first one to finish writing the whole file. \\
Figure \ref{fig:pipeWrite} illustrates the temporal progression of file writing when using pipelining compared to chaining commands with "\texttt{\&\&}".
\begin{figure}[t]
        \import{figures/}{pipeFileWriting.pgf}
    \caption{Using the same settings as in figure \ref{fig:pipeSpeeds} for chaining a \sort with a SAMtools \texttt{view} command, this figure shows the temporal progression of generating each file. The colored bars represent the time in which the output that is annotated at the left axis is generated and written. Thin black lines indicate the file being present on disk but not written to. The time between vertical, dashed lines is used to sort BAM records in memory, the timespan after writing the last file and sorting accounts for reading records from the input file. }
    \label{fig:pipeWrite}
\end{figure}
The diagram shows, that the start of writing the final output, which is the output of SAMtools \texttt{view}, is nearly at the same time as the start of the output of \sort if pipelining is used, but after the full output is written in the case of no pipeline usage. \\
A more realistic use case would be e.g. marking duplicate alignments, which can be done by using 
\begin{minted}{bash} 
samtools fixmate -m example.bam - | \
samtools sort - | \ 
samtools markdup - markdup.bam
\end{minted}
Note that SAMtools commands that require an input and/or output file as parameter must be given "\texttt{-}" if piped. Also, the commands above use unnecessary compression to save on parameters for simplicity. \\

\subsection{Recommendation}
To minimize unnecessary operations and reduce computation time, the parameter \texttt{-u} should be used in the SAMtools sort. This allows for uncompressed output, eliminating the overhead caused by compressing in SAMtools sort and then decompressing immediately afterward with the subsequent command. As shown in \ref{fig:pipeSpeeds}, this reduces the computation time by about one third on for every tested amount of threads.\\
\begin{figure}[t]
        \import{figures/}{pipeSpeeds.pgf}
    \caption{Execution time comparison between different methods of chaining a \texttt{samtools sort} command with a \texttt{samtools view} command: \textit{With temporary file} uses "\texttt{\&\&}" for chaining, both of the others use "\texttt{|}". The command \texttt{samtools sort} utilizes a total of 8\,GB of RAM, while only the RAM parameter (\texttt{-m}), the number of threads, and the \texttt{ -u} flag in the case of \textit{Piped without output compression} are not set to their default values.
    The \texttt{samtools view} command uses default parameters (except number of threads) in every case. SAMtools is compiled with zlib. The input file is a 2.3\,GB unsorted BAM file on the default compression level. With this memory setting, one temporary file is generated.
    }
    \label{fig:pipeSpeeds}
\end{figure}
Exceptions are if the result is not piped to another SAMtools command that reads the output immediately but to an IO operation like  file writing or network transfer. \\
Because of the possibility of exceptions and the difficulty of determining the output destination, removing the compression from SAMtools sort's output on detection of piped output is unreasonable. However, a warning should be displayed, if the output is unspecified. In this case, the filename of the output file is set to "\texttt{-}" and the output is forwarded to standard output using HTSlib. The change in the file name can be detected, and a warning can be printed to standard error, allowing users unfamiliar with compression options to adjust their parameters and save on computation time.


\subsubsection{Prefixes for temporary files} can be set via the "\texttt{-T}" parameter. If no prefix is specified, temporary files are written into the same directory as the output. If the output is to standard output, they are placed in the current working directory. Here, a directory on a fast disk should be chosen. As temporary files are deleted automatically after successful sorting, only at the time of sorting the disks capacities are used. However, it is important to remember, that temporary files are less compressed than regular input and output files. In combination, they require about 20\% more disk space, than the input file. Moreover, if the hard drive is very fast and additionally offers enough storage space, the compression of the intermediate files can be omitted. Unfortunately, this is not possible without changing the source code at the moment. This can be done by simply replacing the parameter \texttt{mode} of the first call of \texttt{bam\_merge\_simple} in the \texttt{bam\_sort\_core\_ext} method which is located in \texttt{bam\_sort.c}. Current values are, depending on the existence of a position too large to be stored in a BAM file, "\texttt{wzx1}" for BGZF compressed SAM files on compression level 1 and "\texttt{wbx1}" for BAM files with compression level 1. Those can be changed to "\texttt{w}" for SAM files and "\texttt{wbx0}" for uncompressed BAM files.
\newpage
\section{Approaches}
\subsection{Storing Pointers}
The initial idea to speed up the sorting process consisted of the following steps: 
\begin{enumerate}
    \item Read once through the whole input file. For every BAM record, store a pointer to the location of the record on the disk together with the attributes needed for sorting, i.e. the reference ID, the position and the REVERSE flag.
    \item Sort the resulting list. As the attributes extracted are much smaller than whole records, this should be possible in memory.
    \item Iterate over the sorted list. For every entry, read the BAM record placed at the location the pointer points to using random reads and write it sequentially into the output file.
\end{enumerate}

Although this method eliminates the need to write intermediate files, which currently consumes a substantial portion of the time needed for sorting, it has some drawbacks: \\
BAM files are binary compressed representations of SAM files. While the compression usually is beneficial to store and transfer the huge amounts of data a SAM file can consist of, it makes random access a lot harder. Usually, a compressed file has to be decompressed from start to at least the position the user is interested in. Especially with the used compression and decompression algorithms, DEFLATE and INFLATE both being streams, every random read would require decompressing the compressed file from the beginning. To solve this, BAM files are compressed using BGZF. As only small blocks are compressed by DEFLATE, for a random read only the number of the block the read is in, together with an offset into the compressed block is needed. \\
However, this method is not suitable for accessing every single record in a file in random order: \\
Blocks typically have sizes of 64KB of uncompressed data. Within our main test file, BAM records had on average a size of about 250 bytes. Therefore, a block on average contains 256 BAM records. To extract every record in random order, the block has to be decompressed 256 times on average to halfway. To make things worse, if the input file is very large in comparison to the available memory, caching  the uncompressed blocks gets less effective. \\
If no merging of temporary files has to be performed, we can now calculate the deflate and inflate operations per BAM record at the current state of SAMtools sort compared to this approach: Currently, the Input file is decompressed once accounting for one INFLATE call for every 265 BAM records. Then, the record is written to the temporary file, resulting in one DEFLATE call for every 256 records. After some time, the temporary file is read again (one INFLATE call per 256 records) and the output file is written (again, one DEFLATE call per 256 records). As the input and the output decompression and compression are necessary for both approaches, they can be ignored. \\
The approach using random reads, however, does not use any DEFLATE call in between input and output, but for every BAM record on average one execution INFLATE on half of a block, accumulating to around 128 INFLATE calls on whole blocks per 256 BAM records. Therefore, to speed up the operation, a single DEFLATE execution (together with writing) would have to be 127 times slower than the combination of reading and INFLATE. As this seems unlikely on most systems, no improvement is expected from this approach. \\
In addition, having to read the file two times breaks the ability of SAMtools sort to work on a stream. As this is a core feature of SAMtools, breaking it should be avoided.

\subsection{Adjust Compression of Temporary Files}
The first measurements showed, that the biggest part of computation time is used for compression even if the compression of the output format is removed. Based on this observation, it seemed obvious, that removing the compression of the temporary files would speed up the process. More experiments on another computer seemed to confirm this hypothesis. However, in final experiments above a threshold of used threads, removing the compression of files turned out to be slower than keeping it on a low level. Where does this change of directions come from? \\
If more or less compression is faster depends on the proportion between compression speed and write speed. The compression speed is mainly influenced by the gzip compression level, the number of available cores, their speed and the implementation of the DEFLATE algorithm. As the output consists of many independly compressed blocks, this can be done in parallel. Thus, increasing the number of threads substantially increases the compression speed if enough physical cores are available. Lowering the compression level also results in a speedup, but as the compression level of temporary files is already 1 the only way to reduce it further would be to disable compression (level 0). Of course, changing to a faster compression library also increases the compression speed. \\
If the write speed the disk offers is less than the compression output, there is an IO bottleneck. To detect this, we can have a look on \textit{IOWait} time. IOWait time is the amount of time, the CPU spends in IDLE because it has to wait for IO Operations. This is measured for all cores together. 

\newpage
\section{Conclusion and Outlook}

In this work, we analyzed \sort for sorting aligned DNA-Read files, specifically BAM files. We found that the most time-consuming part of sorting is compression and writing of output and temporary files. To reduce the runtime of \sort, we proposed setting a higher limit for temporary files concurrently stored on disk, analyzed alternative implementations of the compression library used by SAMtools, and examined the impact of IO requirements on the runtime of \sort.

Setting a higher limit for temporary files concurrently stored on disk reduces the number of merges \sort performs, leading to lower runtimes when sorting large files with limited memory.
By using libdeflate as the compression library, which is automatically the case if SAMtools is installed via Bioconda, a single-thread speedup of 2.3 compared to zlib can be achieved. On 16 threads, this results in a speedup of 1.6.
By utilizing Unix pipelines, we can remove the output compression of \sort, achieving a speedup of 1.8 when \sort is piped to SAMtools \texttt{view} (SAMtools \texttt{view} with zlib compression at level 6).
To increase user awareness of better compression options, we recommended implementing warnings if zlib is used instead of libdeflate, and if the output of \sort is piped but still compressed. 

By using libdeflate for decompression, igzip with compression level 1 for compression of temporary and output files, and an increased limit for temporary files, we could archive a speedup of 6 compared to \sort with the default zlib compression on compression level 6 for single-threaded sorting of a 215\,GiB BAM file utilizing 32\,GiB of memory. Compared to \sort with libdeflate compression on compression level 6, this equals a speedup of 3.\footnote{Runtimes single-threaded: with zlib compression with compression level 6: 41395\,s, with libdeflate compression with compression level 6: 20815\,s, with libdeflate for compression and igzip with compression level 1 for compression: 6887\,s.\\
Runtimes utilizing 16 additional threads: with zlib compression with compression level 6: 4287\,s, with libdeflate compression with compression level 6: 3488\,s, with libdeflate for compression and igzip with compression level 1 for compression: 2217\,s.
}

Utilizing 16 additional threads for sorting of a 215\,GiB BAM file utilizing 32\,GiB of memory, our optimizations lead to a speedup of 2 compared to \sort with the default zlib compression on compression level 6, and a speedup of 1.5 compared to \sort with libdeflate compression on compression level 6.\\

Future projects can investigate the merging process further, as this appears to be a bottleneck for very fast compression libraries. Additionally, they can implement igzip support into SAMtools and its file operation library HTSlib, as igzip has lower runtimes than libdeflate. Furthermore, the merging strategy of \sort can be enhanced by writing temporary files not only to half the limit for temporary files but to the limit minus existing "big files", which are results of previous merges, thereby halving the merges for the first few merges.
\newpage

%
% ---- Bibliography ----
%
% BibTeX users should specify bibliography style 'splncs04'.
% References will then be sorted and formatted in the correct style.
%
\bibliographystyle{splncs04}
\bibliography{refs,references}
\appendix
\appendixpage
\addappheadtotoc
\section{Methods}

\subsection{Setting Compression Levels In SAMtools}\label{methodeComp}
In \sort, the user can set the compression level of the output file using the "\texttt{-l}" parameter. Possible options are integers from 0 to 9. Compression level 0 is equal to no compression (equal to the \texttt{-u} parameter).

For other SAMtools commands where the "\texttt{-l}" parameter does not exist, the user can still change the compression level of the output via adding \texttt{--output-fmt-option level=1} to the arguments of the command (Put the desired compression level between 0 and 9 instead of \texttt{1}).

\subsection{Configuring Libdeflate Support in HTSlib} \label{turnLibdeflate}

To decide manually between using zlib and libdeflate, the user can run the HTSlib \texttt{configure} script with the \texttt{--with-libdeflate} resp. \texttt{--without-libdeflate} option. To use \texttt{LD\_PRELOAD} for changing the zlib implementation, the user must build HTSlib without libdeflate.

\subsection{Libdeflate Compression Level Mapping}\label{compMapping}
\begin{table}[]
    \centering
    \begin{tabular}{l|>{\hspace{0.1em}} c >{\hspace{0.1em}} c >{\hspace{0.1em}} c >{\hspace{0.1em}} c >{\hspace{0.1em}} c >{\hspace{0.1em}} c >{\hspace{0.1em}}c >{\hspace{0.1em}} c >{\hspace{0.1em}} c}
         zlib & \hspace{0.1em} 1 & 2 & 3 & 4 & 5 & \textbf{6} & 7 & 8 & 9 \\
         libdeflate \hspace{0.1em} & \hspace{0.1em} 1 & 2 & 3 & 5 & 6 & \textbf{7} & 8 & 10 & 12 \\
    \end{tabular} \vspace{1em}
    \caption{Mapping between zlib compression levels and libdeflate compression levels in HTSlib. The default level is marked \textbf{bold}.}
    \label{tab:levelMapping}
\end{table}

\subsection{Configuring 7BGZF}\label{7bgzfConfig}

To configure the compression library and the compression level, 7BGZF uses to compress BAM files in the BGZF format, the user can set the \texttt{BGZF\_METHOD} environment variable to a compression library's name concatenated with a compression level before running a SAMtools command. 

For the compression libraries, users can choose one of {\texttt{zlib}}, {\texttt{miniz}}, {\texttt{slz}}, {\texttt{libdeflate}}, {\texttt{zlibng}}, {\texttt{igzip}}, and {\texttt{zopfli}}. Their possible compression levels vary:

\begin{itemize}
\itemsep 0mm
    \item {\texttt{zlib}}, {\texttt{miniz}}, and {\texttt{zlibng}} offer levels from \texttt{1} to \texttt{9}.
    \item {\texttt{libdeflate}} offers levels from \texttt{1} to \texttt{12}.
    \item {\texttt{igzip}} offers levels from \texttt{1} to \texttt{3}.
    \item {\texttt{slz}} only supports level \texttt{1}.
    \item While {\texttt{zopfli}} does not use compression levels in the traditional sense, it allows specifying an amount of iterations (greater than or equal to \texttt{1}) within the compression level parameter of 7BGZF.
\end{itemize}

Example for calling \sort with igzip and compression level 1: \\
\texttt{BGZF\_METHOD=igzip1 LD\_PRELOAD=/path/to/7bgzf.so samtools sort …}


\section{Profiling Timeline Screenshots}

\begin{figure}
    \makebox[\textwidth]{
    \includegraphics[width=1.5\linewidth]{figures/timelinePipe.png}}
\caption{Screenshot of the VTune profiler on running \sort piped to SAMtools \texttt{view}. Both commands utilize 4 threads. However, the total of 8 CPUs the testing machine provides is never reached, as visualized under CPU Utilization. After 5.7 seconds, \sort starts merging the two temporary files it created before into the pipeline. The pipeline's buffer size is 1\,MB and the resulting file has a size of 230\,MB. In the merging and writing phase of \sort, 230 alternating spikes in CPU Time in the workers responsible for compression and decompression of \sort (first 4 rows) and SAMtools \texttt{view} can be found, indicating compression being a bottleneck of the piped operation.} \label{fig1}
\end{figure}



\newpage
\end{document}
