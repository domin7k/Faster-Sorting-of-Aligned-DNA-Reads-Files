% This is samplepaper.tex, a sample chapter demonstrating the
% LLNCS macro package for Springer Computer Science proceedings;
% Version 2.20 of 2017/10/04
%
\documentclass[runningheads]{llncs}
%
\usepackage{minted}
\usepackage{pgf}
\usepackage{import}
\usepackage{layouts}
\usepackage{graphicx}
\usepackage{tikz}
\usetikzlibrary{shapes.geometric, arrows}
\usepackage{amsmath}
\usepackage{adjustbox}

% Used for displaying a sample figure. If possible, figure files should
% be included in EPS format.
%
% If you use the hyperref package, please uncomment the following line
% to display URLs in blue roman font according to Springer's eBook style:
% \renewcommand\UrlFont{\color{blue}\rmfamily}

\renewcommand\#{\protect\scalebox{0.8}{\protect\raisebox{0.4ex}{\char"0023}}}
\definecolor{darkgrey}{RGB}{100,100,100} 

\begin{document}
%
\title{Faster Sorting of Aligned DNA-Reads Files}
%
%\titlerunning{Abbreviated paper title}
% If the paper title is too long for the running head, you can set
% an abbreviated paper title here
%
\author{First Author\inst{1}\orcidID{0000-1111-2222-3333} \and
Second Author\inst{2,3}\orcidID{1111-2222-3333-4444} \and
Third Author\inst{3}\orcidID{2222--3333-4444-5555}}
%
\authorrunning{D. Siebelt}
% First names are abbreviated in the running head.
% If there are more than two authors, 'et al.' is used.
%
\institute{Princeton University, Princeton NJ 08544, USA \and
Springer Heidelberg, Tiergartenstr. 17, 69121 Heidelberg, Germany
\email{lncs@springer.com}\\
\url{http://www.springer.com/gp/computer-science/lncs} \and
ABC Institute, Rupert-Karls-University Heidelberg, Heidelberg, Germany\\
\email{\{abc,lncs\}@uni-heidelberg.de}}
%
\maketitle              % typeset the header of the contribution
%
\begin{abstract}
The abstract should briefly summarize the contents of the paper in
15--250 words.

\keywords{First keyword  \and Second keyword \and Another keyword.}
\end{abstract}
%
%
%
\section{Introduction}
Analysis of aligned DNA-Read data is a crucial part of modern bioinformatics with applications for scientific and medical purposes. DNA-Reads are small sequences of DNA sequences coming off a DNA sequencing machine. They are then aligned to a reference sequence to find variants, disease causing mutations and understand evolutionary relationships between species. In order to analyze specific parts of a genome, such as a chromosome and to locate mutations or variations, DNA-Reads are sorted. This also improves the performance of many other downstream analysis tasks, for example finding and removing duplicate reads. \\

To address the vast amounts of data generated by modern DNA sequencing machines, highly efficient tools and data formats are needed. Developed during the 1000 Genome Project~\cite{the_1000_genomes_project_consortium_1000_2012}, the SAM and BAM formats for storing alignment information to DNA sequences, have found widespread use. Developed alongside these formats, SAMtools became a standard tool for manipulating SAM and BAM files. Next to a collection of many different commands for filtering DNA-Reads, merging of aligned DNA-Read files and many more, SAMtools offers with \sort a tool to alternate the order of the DNA sequences stored in DNA-Read files such as SAM and BAM files. \\

This thesis is about speeding up \sort at sorting DNA-Reads in BAM files with the goal of reducing computation costs and time. BAM is a file format capable of alignment information to DNA sequences in a binary and compressed way. Since DNA data typically occupies substantial storage space, compression is build into the BAM file format in order to reduce storage space and costs, as well as to enable faster transfer over the network.\\

\sort can arrange DNA-Read files in various orders. The default order involves sorting by the reference sequence to which a DNA read is mapped, followed by the position of the mapping on this reference sequence. This enables fast random access to specific regions of interest via index files. \\

Index files are used in the handling of BAM files to facilitate random access despite compression. BAM files are compressed with a special compression method allowing to access the content of the file in blocks without the necessity of decompressing all preceding blocks of the file. The compression format internally uses the compression library zlib for GZIP for compression, which is one of the most popular compression methods. With zlib being open source and used extensively in various applications such as web servers and communication, other libraries have been implemented, offering zlib compatible compression with higher throughput and higher compression efficiency. \\

In this thesis, I present three approaches to speed up the sorting of DNA-Reads stored in BAM files utilizing \sort. \\

\sort performs an external memory sort. If not enough memory is available to \sort, it writes temporary files. If it writes a larger amount of them, it merges them, which is computational intensive. In this thesis, I analyze the impacts of \sort writing and merging temporary files. I propose parameter settings and changes in \sort's internal limitation setting, reducing the amount of merges and thus lowering the computation time of \sort. \\

\sort dedicates a significant amount of its runtime to compressing temporary and output files. To maintain the advantages of compression but reduce its impact on \sort's total runtime, I examine various zlib-compatible compression libraries and the effects of different compression levels. \\

At last, I assess the impacts of IO operations and limitations of IO devices and propose recommendations to minimize or eliminate them.



\subsection{SAMtools}
SAMtools~\cite{12ySamtools} is a collection of tools to work on alignment data, such as aligned DNA-Reads. It relies on the co-developed HTSlib~\cite{bonfield_htslib_2021} for reading and writing information files, e.g. BAM files. SAMtools offers functionality for different operations on alignment data, such as format conversion, statistics, variant calling and many more, including the sorting, which is the focus here. \\

\subsection{Aligned DNA-Reads}
DNA-Reads are sequences of bases coming from a sequencing machine. They can consist of multiple contiguous sequences. Aligned DNA-Reads are DNA-Reads aligned to a reference sequence. The alignment may include insertions, deletions, mismatches, and skipped parts of the reference sequence. Additionally, clipping removes low-quality portions of the sequenced fragment to improve the alignment of the remaining high-quality sequence with the reference. Also, changes in directions on the reference are possible.

\subsection{SAM and BAM files}
A Sequence Alignment Map (SAM) as specified by Li et al.~\cite{samformat} is used to store the alignment of sequences against reference sequences. It consists of a header section and an alignment section. The header section contains meta information such as the format version or the sorting of the content and a dictionary of the reference sequences, whereas the alignment section contains aligned segments with alignment information and meta information such as the read quality. A segment is a continuous sequence or subsequence of a raw DNA-Read. Aligned DNA-Reads are eventually put into multiple records with different segments in the alignment section, as single BAM records can not store changes in directions of the alignment on the reference sequence.\\

The alignment information primarily includes the ID of the reference sequence to which the alignment is mapped, the position in the reference sequence where the alignment starts, and a CIGAR string detailing the alignment at this position. The CIGAR String consists of a list of symbols representing sequential matches, mismatches, insertions, and deletions. Therefore, it represents the alignment of the corresponding segment and its reference sequence. \\

A BAM file is the binary representation of a SAM file. The main differences are the usage of a 4-bit encoding for the sequences, a 3-bit encoding for CIGAR Symbols and a 0-based instead of 1-based coordinate system for the position. Furthermore, a BAM file is per default \textit{BGZF} compressed.

\subsection{BGZF Compression} \label{bgzf}
BGZF, short for Blocked GNU Zip Format, is a lossless compression method proposed at the same time as the BAM format. Widely used compression methods like GZIP compress a file from the beginning to the end in one piece. This has the advantage of allowing matching segments of the file to be located over a greater range. Thus, the compression method is able to reduce the file size more effectively, as repeated sequences can be identified throughout the entire file. However, to decompress such a compressed file, it also needs to be read from the beginning and, depending on the compression method, decompressed at least until the point of interest. \\

Since alignment data can produce very large files but not all of their regions are needed for every use case, it is beneficial to enable some form of random access. To archive this, BGZF utilizes GZIP~\cite{gzip} to compress large files into blocks of less than 64\,KB size (compressed and uncompressed). GZIP uses the DEFLATE algorithm~\cite{deflate} by Phil Katz to compress these individual blocks, which it then subsequently concatenates. Thus, fast random access using index files is possible. In an index file, the position of a piece of information is stored in a 64-bit integer. It consists of a 48-bit unsigned integer \textit{coffset} indicating the number of the compressed block and a 16-bit unsigned integer \textit{uoffset} describing the position in the uncompressed block. \\

Also, the BGZF format provides compatibility with GZIP. Any file compressed utilizing BGZF can be decompressed by any standard GZIP decompression implementation, as GZIP allows this combination of multiple compressed files to one file. Given that Gzip is highly prevalent as a compression technique, there are numerous compatible compression and decompression libraries for all platforms. Thus, employing the open-source GZIP internally simplifies the development of other legacy tools working with BGZF-compatible compression.  \\

Like GZIP, BGZF supports compression levels ranging from 1 (fastest but worst) to 9 (slowest but best) mirroring the compression levels used for the underlying GZIP compression. The compression levels affect the size of the compressed files, as shown in Figure~\ref{fig:compSizes}.
\begin{figure}[t]
        \import{figures/}{compSizes.pgf}
    \caption{Comparison of the size of BGZF compressed files on all compression levels, exemplified using a 10.4\,GiB unsorted BAM file. \\
    Although no compression yields a file four times as large, the distinctions between compression levels are less significant.}
    \label{fig:compSizes}
\end{figure}
The speed of the compression depends mainly on the compression level, the GZIP-implementation and the number of used threads (see Figure \ref{fig:compSpeed}).  
\begin{figure}
        \import{figures/}{compSpeed.pgf}
    \caption{Comparison of the output rate of HTSlib's \texttt{bgzip} which uses BGZF to compress A 10.4GiB unsorted BAM file. For reference, compression level 0 is plotted in the smaller inset plot. \\
    While no compression significantly outpaces every compression level, the throughput increments between the compression levels are consistent.}
    \label{fig:compSpeed}
\end{figure}
To measure the compression speed, I measure the speed of HTSlib's \texttt{bgzip}. This is a tool to compress arbitrary files using BGZF. \sort uses the same methods as \texttt{bgzip} of HTSlib, the library utilized by SAMtools for compression and file operations. This still holds for compression level 0. For this compression level, \texttt{bgzip} as well as \sort do not compress, but directly written the output. Therefore, the compression speed of \texttt{bgzip} is relevant for sorting because it sets a lower boundary on writing the output of \sort, considering that \sort also needs to compress its output.

\subsubsection{Prerequisites}
The process of sorting alternates, depending on some internal Constants and command-line-arguments: \\
We only focus on sorting by the order of the reference, then position and then the REVERSE flag which indicates, if the sequence is aligned forward or backward to the reference. This order is the one used by SAMtools per default, although other sorting criteria e.g. tags or the read name are possible.\\
The maximum amount of memory used to sorting is calculated by the amount of memory the user specifies via the \texttt{-m} option multiplied by the (via \texttt{-@} option) assigned number of threads. Here, we refer to the total amount as \texttt{max\_mem}. \\
The in- and output formats are per default inferred from the file names. \\
SAMtools \texttt{sort} passes its output to standard output if no output file is specified. In this case, the output format is set to BAM.
The maximum number of temporary files is hard-coded as 64 in a constant named \texttt{MAX\_TMP\_FILES}. \\
The gzip compression level for temporary files is set to 1, while the compression level of the result file can be changed via the \texttt{-l} parameter. It defaults to the default compression level used by the library that implements the compression, usually 6, but can be set to a number between 0 (no compression) and 9 (highest and slowest compression).

\subsubsection{Sorting} \label{sorting}

SAMtools performs an external sort process using temporary files that are merged in the end. The sorting process flow is represented by the flowchart in Figure \ref{fig:flow}.
\begin{figure}[ht]
    \begin{adjustbox}{width=\linewidth}
        \tikzstyle{startstop} = [rectangle, rounded corners=5mm, thick,
minimum width=3cm, 
minimum height=1cm,
align=center, 
draw=darkgrey]

\tikzstyle{io} = [trapezium, rounded corners=0.5mm,thick,
trapezium stretches=true, % A later addition
trapezium left angle=70, 
trapezium right angle=110, 
minimum width=3cm, 
minimum height=1cm, 
align=center, 
draw=darkgrey]

\tikzstyle{process} = [rectangle, rounded corners=0.5mm,thick,
minimum width=3cm, 
minimum height=1cm, 
align=center, 
text width=3cm, 
draw=darkgrey]

\tikzstyle{decision} = [diamond, rounded corners=0.5mm, thick,
minimum width=3.2cm, 
minimum height=2.8cm, 
align=center, 
text width=1.9cm,
inner sep=2,
draw=darkgrey]
\tikzstyle{arrow} = [thick,->,>=latex,color=darkgrey]


\begin{tikzpicture}[node distance=2cm]
\tikzstyle{every node}=[font=\footnotesize]

\node (start) [startstop] {Start: \\ \#Files = 0 \\ \#BigFiles = 0\\ \texttt{MAX} := \texttt{MAX\_TMP\_FILES} \\Keep list of current \\temporary files\\};
\node (in1) [io, below of=start, yshift=-0.5cm] {Read BAM records \\ until memory full \\ or EOF};
\node (iseof) [decision, below of=in1, yshift=-0.5cm] {reached EOF?};
\node (pro1) [process, left of=iseof, xshift=-2cm] {Split into Threads blocks and sort them in parallel};
\node (dec1) [decision, below of=pro1, yshift=-1cm] {\vspace{-0.5cm}\\$\#\text{Files} - \#\text{BigFiles} >= \texttt{MAX} / 2$?\vspace{-0.5cm}};
\node (dec2) [decision, right of=dec1, xshift=2cm] {$\#\text{Files} >= \texttt{MAX} $?};
\node (consb) [process, below of=dec1, yshift=-0.5cm] {consolidate\_from := \#BigFiles};
\node (consf) [process, below of=dec2, yshift=-0.5cm] {consolidate\_from := \#Files};
\node (cons0) [process, below of=dec2, right of=dec2, yshift=-0.5cm, xshift=2cm] {consolidate\_from := 0};
\node (merge) [process, below of=consf, text width=13cm] {merge stored files from consolidate\_from to (\#Files - consolidate\_from) and all in
memory files into a file at position \#Files};
\node (consNotNull)[label={[xshift=-0.3mm, text width=2.3cm]center:consolidate\_from\\ \hspace{0.3cm}$>= \#\text{Files}$?}, decision, below of=merge, yshift=-0.5cm] {\phantom{consolidate\_from $>= \#\text{Files}$?}};
\node (updateFs) [process, left of=consNotNull, xshift=-2cm] {Remove merged files,\\ \#Files := \\ consolidate\_from \\ \#BigFiles := consolidate\_from + 1};
\node (addFile) [process, below of=updateFs, yshift=-0.5cm] {\#Files++};
\node(eof) [process, right of=iseof, xshift=2cm] {Sort remaining records in memory using 1 or all Threads depending on amount of records};
\node(end) [startstop, above of=eof, yshift=0.5cm] {Merge all \\ normal, big and \\in-memory files \\ and write final \\output};

\draw [arrow] (start) -- (in1);
\draw [arrow] (in1) -- (iseof);
\draw [arrow] (pro1) -- (dec1);
\draw [arrow] (dec1) -- node[anchor=south, color=black] {no} (dec2);
\draw [arrow] (dec1) -- node[anchor=east, color=black] {yes} (consb);
\draw [arrow] (dec2) -- node[anchor=east, color=black] {yes} (consf);
\draw [color=darkgrey,thick,->,>=latex,rounded corners=2pt] (dec2) -- ++ (3,0) -|  node[pos=0.93,left, color=black] {yes}  (cons0);
\draw [arrow] (cons0) -- (merge);
\draw [arrow] (consf) -- (merge);
\draw [arrow] (consb) -- (merge);
\draw [arrow] (merge) -- (consNotNull);
\draw [arrow] (consNotNull) -- node(yes)[anchor=south, color=black] {yes} (updateFs);
\draw [arrow] (updateFs) -- (addFile);
\draw [color=darkgrey,thick,->,>=latex,rounded corners=2pt] (consNotNull) -- ++ (0,-2) |-  node[pos=0.86,above, color=black] {no}  (addFile);
\draw [color=darkgrey,thick,->,>=latex,rounded corners=2pt] (addFile) -- ++ (-3,0) |-  (in1);
\draw [arrow] (iseof) -- node[anchor=south, color=black] {no} (pro1);
\draw [arrow] (iseof) -- node[anchor=south, color=black] {yes} (eof);
\draw [arrow] (eof) -- (end);

\end{tikzpicture}
    \end{adjustbox}
    \caption{Flow chart showing the current process of sorting, especially the choosing of files to be merged. The list of files is a 0-based list of their names. In the beginning it is empty, after BAM records are read the second time, there is a single record at position 0 and \#Files is 1.}
    \label{fig:flow}
\end{figure}
The sorting starts by sequentially reading BAM records from the input file using HTSlib for parallel decompression. Once the memory limit given by \texttt{max\_mem} is exceeded, these records are split into as many blocks as threads are specified and afterward sorted in parallel. \\
Then, the merge is performed. In the merge, all the sorted in-memory files are written to a single sorted temporary BAM file. In Addition, some of the previously created temporary files are added: The algorithm distinguishes between small files and big files. Small files are files generated by merging one set of in memory blocks. If the number of small files is greater than half of the maximum allowed number of temporary files, all the small files are merged (and afterward deleted). The result of a merge of in-memory and temporary files is a big file. If the total number of files exceeds the limit for temporary files, all temporary files including big files are included in the merge (and afterward deleted). The resulting file is also counted as a big file, despite possibly being much larger than other big files generated by merging only small files. However, as the first merging of big files occurs at the 1120th temporary file\footnote{This number is the result of adding $33 \cdot 33$ temporary files already merged into big files to $31$ small files. Here we have to square $33$, as $32$ small files can exist, and the $33$rd file is the big file which the result of a merge, but not counted among the small files. If $32$ big files exist, there is still space for 32 small files, and they are merged to a $33$rd big file, leaving only space for $31$ small files in the next merging process.}, this is only relevant for combinations of very big files and little memory. \\
In general, the temporary files on the disc can be put into three categories: small files being at most as big as the sorted in-memory blocks together, big files being at most as big as half of the maximum number of allowed temporary files times the maximum size for small files and one big file growing depending on the ratio of allocated memory to the size of the input file possibly to much bigger size than the other big files. \\
After the merge, the algorithm repeats the previous steps until the end of the input file is reached. As the last step, the remaining in-memory BAM records are sorted and merged together with all temporary files and written to the output file.
\section{Failed Approaches}
\subsection{Storing Pointers}
The initial idea to speed up the sorting process consisted of the following steps: 
\begin{enumerate}
    \item Read once through the whole input file. For every BAM record containing alignment information on a DNA-Read, store a pointer to the location of the BAM record on the disk together with the attributes needed for sorting, the ID of the reference sequence, the DNA-Read is aligned to, the starting position of the alignment on this reference sequence and the REVERSE flag.
    \item Sort the resulting list based on the extracted attributes. Due to their smaller memory footprint compared to complete BAM records (10 bytes for the attributes needed for sorting compared to on average 250 bytes for BAM records in our test BAM file), sorting these attributes can be efficiently performed in-memory.
    \item Iterate over the resulting sorted list. For every entry, read the referenced BAM record from disk using random reads and write it sequentially into the output file.
\end{enumerate}

Although this method eliminates the need to write intermediate files, which currently consumes a substantial portion of the time needed for sorting (\Cref{tempfiles}), it has some drawbacks: 

BAM files are binary compressed representations of SAM files containing alignment information of DNA-Reads. While compression is beneficial to store and transfer the huge amounts (up to multiple terabytes per file) of data an aligned DNA-Read file can consist of, it makes random access a lot harder. Usually, a compressed file has to be decompressed from start to at least the position the user is interested in. To address this, BAM files are compressed in the BGZF file format. As a file in the BGZF file format consists of small blocks (less than 64\,KB uncompressed) compressed individually in the  DEFLATE format, for a random read only the number of the block the BAM record for the aligned DNA-Read is in, together with an offset into the compressed block is necessary. 

However, this method is not suitable for accessing every single record in a file in random order:
As mentioned before, the DEFLATE compressed blocks in a BGZF file typically have sizes of 64\,KB of uncompressed data. Within our main test file, BAM records had on average a size of about 250 bytes. Therefore, a DEFLATE compressed block on average contains 256 BAM records. To extract every record in random order, the block has to be decompressed 256 times on average to halfway. Moreover, if the input file is very large in comparison to the available memory, caching  the uncompressed blocks is not feasible. 

We can now approximate the compression and decompression operations per BAM record at the current state of \sort, compared to this approach:\footnote{For simplicity, we ignore possible caching of uncompressed blocks in the approach using random reads.} Currently, \sort decompresses the input file once, accounting for one decompression operation for every 265 BAM records. Then, the record is written to the temporary file, resulting in one compression operation for every 256 records. In the final merge, \sort reads the temporary file again (one compression operation per 256 records) and writes the output file (again, one compression operation for every 256 records). Input and output decompression and compression are necessary for both approaches, therefore they account for the same amount of compression and decompression operations in both approaches.

The approach using random reads, however, does not use compression operations in between reading the input file and writing the output file, but for every BAM record on average one compression operation on half a block, accumulating to around 128 decompression operations on whole blocks per 256 BAM records. Therefore, for the approach using random reads to be faster than the current behavior of \sort, compression of a single block (together with writing) would have to be 127 times slower than the combination of reading and decompressing a compressed block. However, with the default zlib compression library, compression on compression level 6 is approximately 9 times slower than decompression of a file. 

These considerations apply only to sorting without temporary files. In the case where \sort utilizes temporary files, each write operation would add one compression and decompression step for each block of BAM records. These considerations are not supported by experiments. However, due to the multiple decompression operations for each block, a speedup from this approach seems unlikely.

In addition, having to read the file two times breaks the ability of SAMtools sort to work on a stream. As this is a core feature of SAMtools, breaking it should be avoided.

\subsection{Removing Compression of Temporary Files}
Our first measurements showed, that \sort spends most of its computation time for compression, even if it outputs uncompressed aligned DNA-Read files. Based on this observation, our initial assumption was that removing the compression of the temporary files would reduce the runtime of \sort. However, experimenting with faster compression libraries and utilizing a larger amount of CPUs, removing the compression of files turned out to be slower than keeping it on a low level (\Cref{ioComp}). Therefore, we decided not to change the compression \sort applies to temporary files. 
\section{Pipelines}

\subsection{Unix Pipelines}

\textit{Pipelines} are a way of forwarding the output of one program to the input of another program. In Unix-like operating systems, it connects the standard output of the first of one program with the standard input of another program. This is implemented by a ring-buffer in the memory and usually expressed in the shell by writing a vertical bar "\texttt{|}" between the commands. For example, 
\begin{minted}{bash} 
ls | grep .bam 
\end{minted}
lists all files in the current directory using \texttt{ls} and then filters them using \texttt{grep} to display only files having ".bam" in their name. 

\subsection{Pipelining in SAMtools}
In the context of SAMtools, piping can be used to chain multiple commands without the need to write temporary files. This can result in a huge speedup, as neither compression is needed if pipes are used nor the temporary output has to be written to the disk. These two operations account for a significant portion of the runtime of SAMtools \texttt{sort} command. In addition, as the outputs are streams, the second command can start processing the output of the first one as soon, as it begins to be generated instead of having to wait for the first one to finish writing the whole file.
Marking duplicate alignments for example can be done by using 
\begin{minted}{bash} 
samtools fixmate -m example.bam - | \
samtools sort - | \ 
samtools markdup - markdup.bam
\end{minted}
Note that SAMtools commands that require an input and/or output file as parameter must be given "\texttt{-}" if piped. Also, the commands above use unnecessary compression to save on parameters for simplicity. \\

\subsection{Recommendation}
To minimize unnecessary operations and shorten computation time, the parameter \texttt{-u} should be used in SAMtools sort. This allows for uncompressed output, eliminating the overhead caused by compressing in SAMtools sort and then decompressing immediately afterward with the subsequent command. \\
Exceptions are if the result is not piped to another SAMtools command that reads the output immediately but to an IO operation like  file writing or network transfer. \\
Because of the possibility of exceptions and the difficulty of determining the output destination, removing the compression from SAMtools sort's output on detection of piped output is unreasonable. However, a warning should be displayed, if the output is unspecified. In this case, the filename of the output file is set to "\texttt{-}" and the output is forwarded to standard output using HTSlib. The change in the file name can be detected, and a warning can be printed to standard error, allowing users unfamiliar with compression options to adjust their parameters and save on computation time.

\begin{figure}
    \begin{center}
        \import{figures/}{pipeSpeeds.pgf}
    \end{center}
    \caption{Execution time comparison between different methods of chaining a \texttt{samtools sort} command with a \texttt{samtools view} command: \textit{With temporary file} uses "\texttt{\&\&}" for chaining, both of the others use "\texttt{|}". The \texttt{samtools sort} command utilizes a total of 8GB of RAM, while only the RAM parameter (\texttt{-m}), the number of threads, and the \texttt{-u} flag in the case of \textit{Piped without output compression} are not set to their default values.
    The \texttt{samtools view} command uses default parameters (except number of threads) in every case. SAMtools is compiled with zlib. The Input file is a 2.3GB unsorted BAM file on default compression level. With this memory settings, one temporary file is generated.
    }
\end{figure}






%
% ---- Bibliography ----
%
% BibTeX users should specify bibliography style 'splncs04'.
% References will then be sorted and formatted in the correct style.
%
\bibliographystyle{splncs04}
\bibliography{refs,references}

\end{document}
