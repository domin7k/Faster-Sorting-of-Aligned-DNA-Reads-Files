\section{Pipelines}

\subsection{Unix Pipelines}

\textit{Pipelines} are a way of forwarding the output of one program to the input of another program. In Unix-like operating systems, it connects the standard output of the first of one program with the standard input of another program. This is implemented by a ring buffer in the memory and is usually expressed in the shell by writing a vertical bar "\texttt{|}" between the commands. For example, 
\begin{minted}{bash} 
ls | grep .bam 
\end{minted}
lists all files in the current directory using \texttt{ls} and then filters them using \texttt{grep} to display only files having ".bam" in their name. 

\subsection{Pipelining in SAMtools}
In the context of SAMtools, pipelining can be used to chain multiple commands without the need to write temporary files. This can result in a huge speedup, as neither compression is needed if pipes are used nor the temporary output has to be written to the disk. These two operations account for a significant portion of the runtime of the SAMtools \texttt{sort} command. In addition, as the outputs are streams, the second command can start processing the output of the first one as soon as it begins to be generated, instead of having to wait for the first one to finish writing the whole file.
Marking duplicate alignments for example can be done by using 
\begin{minted}{bash} 
samtools fixmate -m example.bam - | \
samtools sort - | \ 
samtools markdup - markdup.bam
\end{minted}
Note that SAMtools commands that require an input and/or output file as parameter must be given "\texttt{-}" if piped. Also, the commands above use unnecessary compression to save on parameters for simplicity. \\

\subsection{Recommendation}
To minimize unnecessary operations and reduce computation time, the parameter \texttt{-u} should be used in the SAMtools sort. This allows for uncompressed output, eliminating the overhead caused by compressing in SAMtools sort and then decompressing immediately afterward with the subsequent command. As shown in \ref{fig:pipeSpeeds}, this reduces the computation time by about one third on for every tested amount of threads.\\
\begin{figure}

    \begin{center}
        \import{figures/}{pipeSpeeds.pgf}
    \end{center}
    \caption{Execution time comparison between different methods of chaining a \texttt{samtools sort} command with a \texttt{samtools view} command: \textit{With temporary file} uses "\texttt{\&\&}" for chaining, both of the others use "\texttt{|}". The command \texttt{samtools sort} utilizes a total of 8GB of RAM, while only the RAM parameter (\texttt{-m}), the number of threads and the \texttt{ -u} flag in the case of \textit{Piped without output compression} are not set to their default values.
    The \texttt{samtools view} command uses default parameters (except number of threads) in every case. SAMtools is compiled with zlib. The input file is a 2.3GB unsorted BAM file on the default compression level. With this memory setting, one temporary file is generated.
    }
    \label{fig:pipeSpeeds}
\end{figure}
Exceptions are if the result is not piped to another SAMtools command that reads the output immediately but to an IO operation like  file writing or network transfer. \\
Because of the possibility of exceptions and the difficulty of determining the output destination, removing the compression from SAMtools sort's output on detection of piped output is unreasonable. However, a warning should be displayed, if the output is unspecified. In this case, the filename of the output file is set to "\texttt{-}" and the output is forwarded to standard output using HTSlib. The change in the file name can be detected, and a warning can be printed to standard error, allowing users unfamiliar with compression options to adjust their parameters and save on computation time.





