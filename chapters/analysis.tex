\section{Analysis (Version 1.19.2)}

\subsection{Algorithm}
\subsubsection{Prerequisites}
The process of sorting alternates, depending on some internal Constants and command-line-arguments: \\
We only focus on sorting by the order of the reference, then position and then the REVERSE flag which indicates, if the sequence is aligned forward or backward to the reference. This order is the one used by SAMtools per default, although other sorting criteria e.g. tags or the read name are possible.\\
The maximum amount of memory used to sorting is calculated by the amount of memory the user specifies via the \texttt{-m} option multiplied by the (via \texttt{-@} option) assigned number of threads. Here, we refer to the total amount as \texttt{max\_mem}. \\
The in- and output formats are per default inferred from the file names. \\
SAMtools \texttt{sort} passes its output to standard output if no output file is specified. In this case, the output format is set to BAM.
The maximum number of temporary files is hard-coded as 64 in a constant named \texttt{MAX\_TMP\_FILES}. \\
The gzip compression level for temporary files is set to 1, while the compression level of the result file can be changed via the \texttt{-l} parameter. It defaults to the default compression level used by the library that implements the compression, usually 6, but can be set to a number between 0 (no compression) and 9 (highest and slowest compression).

\subsubsection{Sorting} \label{sorting}

SAMtools performs an external sort process using temporary files that are merged in the end. The sorting process flow is represented by the flowchart in Figure \ref{fig:flow}.
\begin{figure}[ht]
    \begin{adjustbox}{width=\linewidth}
        \tikzstyle{startstop} = [rectangle, rounded corners=5mm, thick,
minimum width=3cm, 
minimum height=1cm,
align=center, 
draw=darkgrey]

\tikzstyle{io} = [trapezium, rounded corners=0.5mm,thick,
trapezium stretches=true, % A later addition
trapezium left angle=70, 
trapezium right angle=110, 
minimum width=3cm, 
minimum height=1cm, 
align=center, 
draw=darkgrey]

\tikzstyle{process} = [rectangle, rounded corners=0.5mm,thick,
minimum width=3cm, 
minimum height=1cm, 
align=center, 
text width=3cm, 
draw=darkgrey]

\tikzstyle{decision} = [diamond, rounded corners=0.5mm, thick,
minimum width=3.2cm, 
minimum height=2.8cm, 
align=center, 
text width=1.9cm,
inner sep=2,
draw=darkgrey]
\tikzstyle{arrow} = [thick,->,>=latex,color=darkgrey]


\begin{tikzpicture}[node distance=2cm]
\tikzstyle{every node}=[font=\footnotesize]

\node (start) [startstop] {Start: \\ \#Files = 0 \\ \#BigFiles = 0\\ \texttt{MAX} := \texttt{MAX\_TMP\_FILES} \\Keep list of current \\temporary files\\};
\node (in1) [io, below of=start, yshift=-0.5cm] {Read BAM records \\ until memory full \\ or EOF};
\node (iseof) [decision, below of=in1, yshift=-0.5cm] {reached EOF?};
\node (pro1) [process, left of=iseof, xshift=-2cm] {Split into Threads blocks and sort them in parallel};
\node (dec1) [decision, below of=pro1, yshift=-1cm] {\vspace{-0.5cm}\\$\#\text{Files} - \#\text{BigFiles} >= \texttt{MAX} / 2$?\vspace{-0.5cm}};
\node (dec2) [decision, right of=dec1, xshift=2cm] {$\#\text{Files} >= \texttt{MAX} $?};
\node (consb) [process, below of=dec1, yshift=-0.5cm] {consolidate\_from := \#BigFiles};
\node (consf) [process, below of=dec2, yshift=-0.5cm] {consolidate\_from := \#Files};
\node (cons0) [process, below of=dec2, right of=dec2, yshift=-0.5cm, xshift=2cm] {consolidate\_from := 0};
\node (merge) [process, below of=consf, text width=13cm] {merge stored files from consolidate\_from to (\#Files - consolidate\_from) and all in
memory files into a file at position \#Files};
\node (consNotNull)[label={[xshift=-0.3mm, text width=2.3cm]center:consolidate\_from\\ \hspace{0.3cm}$>= \#\text{Files}$?}, decision, below of=merge, yshift=-0.5cm] {\phantom{consolidate\_from $>= \#\text{Files}$?}};
\node (updateFs) [process, left of=consNotNull, xshift=-2cm] {Remove merged files,\\ \#Files := \\ consolidate\_from \\ \#BigFiles := consolidate\_from + 1};
\node (addFile) [process, below of=updateFs, yshift=-0.5cm] {\#Files++};
\node(eof) [process, right of=iseof, xshift=2cm] {Sort remaining records in memory using 1 or all Threads depending on amount of records};
\node(end) [startstop, above of=eof, yshift=0.5cm] {Merge all \\ normal, big and \\in-memory files \\ and write final \\output};

\draw [arrow] (start) -- (in1);
\draw [arrow] (in1) -- (iseof);
\draw [arrow] (pro1) -- (dec1);
\draw [arrow] (dec1) -- node[anchor=south, color=black] {no} (dec2);
\draw [arrow] (dec1) -- node[anchor=east, color=black] {yes} (consb);
\draw [arrow] (dec2) -- node[anchor=east, color=black] {yes} (consf);
\draw [color=darkgrey,thick,->,>=latex,rounded corners=2pt] (dec2) -- ++ (3,0) -|  node[pos=0.93,left, color=black] {yes}  (cons0);
\draw [arrow] (cons0) -- (merge);
\draw [arrow] (consf) -- (merge);
\draw [arrow] (consb) -- (merge);
\draw [arrow] (merge) -- (consNotNull);
\draw [arrow] (consNotNull) -- node(yes)[anchor=south, color=black] {yes} (updateFs);
\draw [arrow] (updateFs) -- (addFile);
\draw [color=darkgrey,thick,->,>=latex,rounded corners=2pt] (consNotNull) -- ++ (0,-2) |-  node[pos=0.86,above, color=black] {no}  (addFile);
\draw [color=darkgrey,thick,->,>=latex,rounded corners=2pt] (addFile) -- ++ (-3,0) |-  (in1);
\draw [arrow] (iseof) -- node[anchor=south, color=black] {no} (pro1);
\draw [arrow] (iseof) -- node[anchor=south, color=black] {yes} (eof);
\draw [arrow] (eof) -- (end);

\end{tikzpicture}
    \end{adjustbox}
    \caption{Flow chart showing the current process of sorting, especially the choosing of files to be merged. The list of files is a 0-based list of their names. In the beginning it is empty, after BAM records are read the second time, there is a single record at position 0 and \#Files is 1.}
    \label{fig:flow}
\end{figure}
The sorting starts by sequentially reading BAM records from the input file using HTSlib for parallel decompression. Once the memory limit given by \texttt{max\_mem} is exceeded, these records are split into as many blocks as threads are specified and afterward sorted in parallel. \\
Then, the merge is performed. In the merge, all the sorted in-memory files are written to a single sorted temporary BAM file. In Addition, some of the previously created temporary files are added: The algorithm distinguishes between small files and big files. Small files are files generated by merging one set of in memory blocks. If the number of small files is greater than half of the maximum allowed number of temporary files, all the small files are merged (and afterward deleted). The result of a merge of in-memory and temporary files is a big file. If the total number of files exceeds the limit for temporary files, all temporary files including big files are included in the merge (and afterward deleted). The resulting file is also counted as a big file, despite possibly being much larger than other big files generated by merging only small files. However, as the first merging of big files occurs at the 1120th temporary file\footnote{This number is the result of adding $33 \cdot 33$ temporary files already merged into big files to $31$ small files. Here we have to square $33$, as $32$ small files can exist, and the $33$rd file is the big file which the result of a merge, but not counted among the small files. If $32$ big files exist, there is still space for 32 small files, and they are merged to a $33$rd big file, leaving only space for $31$ small files in the next merging process.}, this is only relevant for combinations of very big files and little memory. \\
In general, the temporary files on the disc can be put into three categories: small files being at most as big as the sorted in-memory blocks together, big files being at most as big as half of the maximum number of allowed temporary files times the maximum size for small files and one big file growing depending on the ratio of allocated memory to the size of the input file possibly to much bigger size than the other big files. \\
After the merge, the algorithm repeats the previous steps until the end of the input file is reached. As the last step, the remaining in-memory BAM records are sorted and merged together with all temporary files and written to the output file.
\FloatBarrier

\subsection{Time Allocation}
Understanding the resource utilization and the time allocation of the different parts of the sorting process is crucial to be able to optimize its computation time. However, the process has different points of constraint on different machines, as we will see in the following. \\
In general, high time consumption of the SAMtools \texttt{sort} method can be traced to three main blocks.

\subsubsection{Compression} is a part of writing BAM files, as per default compression is applied to all BAM files and even part of the specification. Although compression of BAM files is beneficial in the long term in order to reduce storing costs and transfer speed, it comes with a significant resource overhead. \\
Performing SAMtools \texttt{sort} on a laptop, trough various settings compression and decompression together account for around 95\% of the CPU time. Approximately 80\% are solely required by the \textit{deflate} method that is responsible for the compression.

\subsubsection{IO} can also be a constraint of the sorting process. As the internal mechanisms of SAMtools usually work very fast and are highly parallel, but have to process huge amounts of data, input and output devices can also limit the computation speed.

\subsubsection{Temporary Files} are necessary for SAMtools sort to work as a stream while processing more data than can be held in memory. Unfortunately, writing temporary files is time-consuming. When only looking at the time between reading and decompressing the input and writing and compressing the output, operations involving temporary files lead to the most time consumption. Thus, the amount of temporary files should be minimized. More specific, a BAM record should be written as infrequently as possible. On the other hand, limitations of the Operating System have to be taken into consideration.

\subsection{Compression}
HTSlib is the tool used by SAMtools to perform all file operations. On its README, it claims its only dependency to be \textit{zlib}. zlib is a library used for compression utilizing the DEFLATE algorithm. 

\subsection{IO}


\subsection{Temporary Files}
SAMtools \texttt{sort} has, as mentioned above, a hard coded limit for temporary files. Moreover, this limit is reached very late because of multi level merging. \\
To understand how many temporary files are written, we have to look into the algorithm for merging. The first variable influencing the generation of temporary files is the memory limit. Defaulting to 768MiB, it gets multiplied by the number of threads. The result is the limit up to which BAM records are read in one block. This is also a good approximation for the size of a small temporary file before compression. At least one MiB per thread is enforced to prevent the creation of a huge amount of temporary files. Intuitively, one would think, that the sorting gets faster the more memory can be used. Figure \ref{fig:maxMems} illustrates that this is generally the case, although not in a linear proportion.
\begin{figure}
        \import{figures/}{maxMems.pgf}
    \caption{Execution time of SAMtools \texttt{sort} on a 2.4GB BAM file using default parameters except \texttt{-m} for memory limitation setting. }
    \label{fig:maxMems}
\end{figure}
Moreover, between 400MiB and 12800MiB memory allocation the execution time increases - despite using up to 32 times more memory. To investigate further, we can take a look at the amount of temporary files produced. The input file expands to just a little bigger than the second-highest memory limitation in Figure \ref{fig:maxMems}. Therefore, at the highest setting 25600MiB which equals to 25GiB, no temporary file is produced. On the next highest settings, 1, 2, 4, ... temporary files are produced, as the \texttt{max\_mem} parameter halves to every next highest value. Looking at the amount of temporary files produced, we can also approximate the size of the BAM file in memory. At 400MiB, 32 temporary files are generated as expected. At 200MiB, 65 temporary files are generated. This indicates, that after having processed 12800MiB of data, 200MiB are not enough to keep the remaining data in memory until the final merge into the output file, but 400MiB are. For this reason, the size of the BAM file in memory increases to between 13000MiB and 13200MiB. \\
Now we can see why there are no speed improvements between 400MiB and 12800MiB. In between those settings, exactly the same records are written to the disk, the only difference is the number of files they are spit in. \\
This changes at 200MiB \texttt{max\_mem}. The total of 65 produced temporary files means, that one merge is performed before the final merge and big file is generated. This comes with additional time consumption because the content of the first 32 files has to be read from disk, decompressed, merged, compressed  and written to disk again a second time. \\
At 100MiB 3 big files are generated, at 50MiB 7 and at 25MiB 15. This is also reflected in the total amount of bytes written. While in the parameter settings producing temporary files but not enough of them to be merged to big files 2.4GiB in temporary files are written, this number goes up to 3.7GiB, 4.3Gi, 4.6GiB and 4.8 GiB for 200MiB, 100MiB, 50MiB and 25MiB. Here, the increase in total written bytes for temporary files is not proportional to the amount of merges, as the size of the merged files shrinks with lowering the \texttt{max\_mem} parameter. In Addition, the proportional influence on the total time spend before merging the final result gets lower with the number of performed merges: While writing the first big temporary file costs approximately as much as writing all temporary files before, writing the second one costs only a third of all file writing before, the next one 1/5 then 1/7 and so on.\\
Obviously, the measurements above are unrealistic, as nowadays even Laptops have more memory installed. At the same time, BAM files are usually way bigger than the used sample, which was actually sampled by randomly taking 1\% of a real world BAM file. To get an impression of the impacts of increasing file size, we can look at what changes. \\
Both compression and decompression work in $\mathcal{O}(n)$, ensured by the blockwise compression. The sorting method used is a radix sort, which is also in $\mathcal{O}(n)$. For merging, a heap based approach is chosen, which works in $\mathcal{O}(n \log(k))$. Here, $k$ is the number of sorted list to be merged. Thus, in theory, keeping the same ratio of input size and available memory should produce the same amount of temporary files and in memory files which are the result of sorting and also included in the merging process. Together with all other operations being in linear time, results on little files with little memory should transfer proportional to big files and more memory. This is confirmed by the experiment in Figure \\
\begin{figure}
        \import{figures/}{maxMems.pgf}
    \caption{Execution time of SAMtools \texttt{sort} on a 2.4GB BAM file using default parameters except \texttt{-m} for memory limitation setting. }
    \label{fig:scaleMem}
\end{figure}
However, changing only one of these parameters has different effects. Using SAMtools for example locally installed on a laptop to sort a bigger BAM file can produce many temporary files if memory is limited. If e.g. 8GB are available for SAMtools \texttt{sort}, files bigger than 50GB can not be sorted without merging temporary files. This behavior gets worse if the ratio of the input file to the \texttt{max\_mem} setting grows further.