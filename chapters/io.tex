\section{Input/Output} 
Input and Output can also be constraints of the sorting process. As the internal mechanisms of SAMtools usually work very fast and are highly parallel, but have to process huge amounts of data, input and output devices can also limit the computation speed. \\
In contrast to a program's behavior or used libraries, often the IO devices can hardly be changed. However, in some cases there are possibilities to speed up the process.

\subsection{Unix Pipelines}

\textit{Pipelines} are a way of forwarding the output of one program to the input of another program. In Unix-like operating systems, they connect the standard output of one program with the standard input of another program. In the Linux Kernel, this is implemented \cite{noauthor_linuxfspipec_nodate} by a ring buffer in the memory and is usually expressed in the shell by writing a vertical bar "\texttt{|}" between the commands. For example, 
\begin{minted}{bash} 
ls | grep .bam 
\end{minted}
lists all files in the current directory using \texttt{ls} and then filters them using \texttt{grep} to display only files having ".bam" in their name. 

\subsection{Pipelining in SAMtools}
In the context of SAMtools, pipelining can be used to chain multiple commands without the need to write temporary files. This can result in a huge speedup, as neither compression is needed if pipes are used nor the temporary output has to be written to the disk. These two operations account for a significant portion of the runtime of the SAMtools \texttt{sort} command. In addition, as the outputs are streams, the second command can start processing the output of the first one as soon as it begins to be generated, instead of having to wait for the first one to finish writing the whole file. \\
Figure \ref{fig:pipeWrite} illustrates the temporal progression of file writing when using pipelining compared to chaining commands with "\texttt{\&\&}".
\begin{figure}
        \import{figures/}{pipeFileWriting.pgf}
    \caption{Using the same settings as in figure \ref{fig:pipeSpeeds} for chaining a \texttt{samtools sort} with a \texttt{samtools view} command, this figure shows the temporal progression of generating each file. The colored bars represent the time in which the output that is annotated at the left axis is generated and written. Thin black lines indicate the file being present on disk but not written to. The time between vertical, dashed lines is used to sort BAM records in memory, the timespan after writing the last file and sorting accounts for reading records from the input file. The diagram shows, that the start of writing the final output, which is the output of \texttt{samtools view}, is nearly at the same time as the start of the output of \texttt{samtools sort} if pipelining is used, but after the full output is written in the case of no pipeline usage.}
    \label{fig:pipeWrite}
\end{figure}
A more realistic use case would be e.g. marking duplicate alignments, which can be done by using 
\begin{minted}{bash} 
samtools fixmate -m example.bam - | \
samtools sort - | \ 
samtools markdup - markdup.bam
\end{minted}
Note that SAMtools commands that require an input and/or output file as parameter must be given "\texttt{-}" if piped. Also, the commands above use unnecessary compression to save on parameters for simplicity. \\

\subsection{Recommendation}
To minimize unnecessary operations and reduce computation time, the parameter \texttt{-u} should be used in the SAMtools sort. This allows for uncompressed output, eliminating the overhead caused by compressing in SAMtools sort and then decompressing immediately afterward with the subsequent command. As shown in \ref{fig:pipeSpeeds}, this reduces the computation time by about one third on for every tested amount of threads.\\
\begin{figure}
        \import{figures/}{pipeSpeeds.pgf}
    \caption{Execution time comparison between different methods of chaining a \texttt{samtools sort} command with a \texttt{samtools view} command: \textit{With temporary file} uses "\texttt{\&\&}" for chaining, both of the others use "\texttt{|}". The command \texttt{samtools sort} utilizes a total of 8GB of RAM, while only the RAM parameter (\texttt{-m}), the number of threads, and the \texttt{ -u} flag in the case of \textit{Piped without output compression} are not set to their default values.
    The \texttt{samtools view} command uses default parameters (except number of threads) in every case. SAMtools is compiled with zlib. The input file is a 2.3GB unsorted BAM file on the default compression level. With this memory setting, one temporary file is generated.
    }
    \label{fig:pipeSpeeds}
\end{figure}
Exceptions are if the result is not piped to another SAMtools command that reads the output immediately but to an IO operation like  file writing or network transfer. \\
Because of the possibility of exceptions and the difficulty of determining the output destination, removing the compression from SAMtools sort's output on detection of piped output is unreasonable. However, a warning should be displayed, if the output is unspecified. In this case, the filename of the output file is set to "\texttt{-}" and the output is forwarded to standard output using HTSlib. The change in the file name can be detected, and a warning can be printed to standard error, allowing users unfamiliar with compression options to adjust their parameters and save on computation time.


\subsubsection{Pipelining} enables all advantages of working with streams. For SAMtools \texttt{sort}, this has the following consequences: Records can be read as they are produced by the previous command. This can be done without compression and without having to write files to the disk using UNIX pipelines. Note that the process generating the input for SAMtools \texttt{sort} is most likely halted once the memory limit of \texttt{sort} is reached. This is due to the buffer offered by the pipeline being full and only emptied again, after a temporary files is written. Looking at the output of sort, there is also no need for compression and writing to disk if piped. 

\subsubsection{Prefixes for temporary files} can be set via the "\texttt{-T}" parameter. If no prefix is specified, temporary files are written into the same directory as the output. If the output is to standard output, they are placed in the current working directory. Here, a directory on a fast disk should be chosen. As temporary files are deleted automatically after successful sorting, only at the time of sorting the disks capacities are used. However, it is important to remember, that temporary files are less compressed than regular input and output files. In combination, they require about 20\% more disk space, than the input file. Moreover, if the hard drive is very fast and additionally offers enough storage space, the compression of the intermediate files can be omitted. Unfortunately, this is not possible without changing the source code at the moment. This can be done by simply replacing the parameter \texttt{mode} of the first call of \texttt{bam\_merge\_simple} in the \texttt{bam\_sort\_core\_ext} method which is located in \texttt{bam\_sort.c}. Current values are, depending on the existence of a position too large to be stored in a BAM file, "\texttt{wzx1}" for BGZF compressed SAM files on compression level 1 and "\texttt{wbx1}" for BAM files with compression level 1. Those can be changed to "\texttt{w}" for SAM files and "\texttt{wbx0}" for uncompressed BAM files.